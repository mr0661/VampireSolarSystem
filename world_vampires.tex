%Tässä puhutaan player = ei-pelinjohtaja pelaaja
This chapter is about creating vampire mythos through play.
It might be that you do not have any specific setting in mind and don't want to do research to find correct setting for you.
Perhaps you don't want to force some setting or are interested of creating your own.
This chapter might be just right for you.

Vampires have three components.
They have social component, strengths, and weaknesses.
Social component are shown as natural consequence of the play.
When vampire players interact, that shows how vampires socialize.
Strengths and weaknesses is different matter.
We will delve little deeper into those.

First golden rule is, \textit{if all players agree}, it goes.
When all player agree that something is relevant property of vampire, then it is.
No futher mechanic is needed, though player might want to write it down.
This is very important part, which doe its simplicity does not get much hilighted.
Other ways of inventing new flavor to vampires are little more complex, so they take more space.
But before thinking of applying any other rule, first consider if all players already agree.

Second concept is, that each player is allowed to add as much as they want to the lore.
This books contains examples what to add, but should not be considered as limiting factor.
Only limitation is that these should be choses made these choices as a player, with responisibility related to that.
This means, that this addition is made with intention to make game better.

Lastly, no addition should be made to cancel previously made decisions.
Simply put, when adding new content, it should add, not replace or remove.
This is to prevent nullifying other players intentions.
This rule is difficult to follow, and mistakes can happen.
Simple case for this is if first player adds that vampires start sparkle on moonlight, revealing their presence, 
	second player should not add that vampires can turn invisible in moonlight.
Note that some vampires might have special powers that nullify these curses, 
	but remember that this chapter is about vampires in general.

\subsubsection{Strengths}
This means that learning about uses of this stregth is focal point for the character.
It is represented with related key (drunk of power, OR unable to control).
After this key is bought, the related power is immediatly added part of the vampiric secret which gives benefit for all the characters with that secret.
For player character adding the power, there is related key, that produces some story relating to this power.
If you have characters that are not vampires, they can also evoke this, they just don't get the benefits.

\subsubsection{Weaknesses}
This means that learning about weakness, or having it just start affecting.
This experience is represented with related key (weakness found).
After this key is bought, the related weakness is immediatly added part of the vampiric secret which gives disadvantages for all the characters with that secret.
For player character adding the weakness, there is related key, that produces some story relating to this power.
If you have characters that are not vampires, they can also evoke this, they just don't get the effects,
	Key bought should still be relevant.

\pagebreak
