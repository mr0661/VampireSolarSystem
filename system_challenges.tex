Challenge is core resolution mechanism.
It is used to determine whether action relevant to the story succeeds or fails.
There is two types of challenges, simpler case is uncontested and other one is conflict (Section \ref{ssec:conflict}).
This chapter focuses on uncontested and any challenge in general.

\subsubsection{Base die}

Game uses fudge die.
These are six sided dice with three different sides: '-', ' ', and '+'.
Easy way to use any die as fudge die is to say that minus is 1 and 2, empty is 3 and 4, and plus is 5 an 6.

When story moves into situation, where success of character is questioned, challenge is called.
\GM{} and protagonists player should first discuss on stakes on the roll.
This is somewhat atypical for RPGs, it should be clear what are stakes when any roll is made.
This is decided by freely discussing what the attempt means in fiction.
It is up to the player to choose how much is resolved by any single roll, but stakes should match.
\GM{} is allowed to determine that if the protagonist would try it, it would automatically fail.

When challenge is called, player of the character throws three dice.
Player selects proper ability to use for the roll.
That ability score serves base result.
That is modified by total of the roll, each '-' decreases number by one, and each '+' increases number by one (empty do not have an effect).
Total value determines the result.
This can be intepreted in following way:
\begin{center}
\begin{tabular}{ c c l }
 \textbf{Total} & \textbf{Result} & \textbf{Description}\\ 
 0- & Failure & Character fails and will face consequences \\
 1 & Marginal & Success without any glamour \\
 2 & Good & Character is showing that they are good at what they do \\
 3 & Great & Something to be proud of \\
 4 & Amazing & Witnesses are impressed \\
 5 & Legendary & This act will inspire stories \\
 6 & Ultimate & Perfect execution, nobody can see any flaw in the act \\
 7 & Trascendent & World is permanently altered
\end{tabular}
\end{center}
% TODO Next part is copied
\textit{A Failure (0)} result does not have to mean that the character’s efforts just fizzle; 
	most of the time failure in the Solar System should be a cinematic affair full of dramatic consequence. 
A character who fails in a Climbing (V) check isn’t going to just “not get on the top of the cliff”, 
	he’s going to fall down from high and break his collarbone, being stranded alone and wounded in the wilderness.

\textit{A Marginal (1)} success might sound dull and predictable, but what it actually means is that the Story Guide is welcome to add a little twist, some complication to the success. 
The character might indeed get to the top of the cliff, but it might take so long that night falls during the climb, for example.

What characters really want is a \textit{Good (2)}, solid success, as that’s something to be proud of. 
Colleagues won’t find flaws in it, it’s the real thing. 
A bit dull, perhaps, but secure, as nobody’s going to interpret it as anything but solid fulfillment of the stakes.

\textit{A Great (3)} success is essentially master-level, it’s good and a little bit more, yet. 
The player would be well within his rights to describe how his character not only succeeds, but also does it in style. 
Getting to the top of the cliff is a given, but the character might as well find an easy route up, one that makes a second climb nigh trivial for him later on.

\textit{An Amazing (4)} success, as the chart says, is something that onlookers would be astounded by. 
A character making that kind of check deserves to have a solid edge further in the scene
	 — finding a sheltered cave in the cliffside to spend the night could be a positive twist supplied by the Story Guide, for example. 
Not exactly what the character tried to do, but a further positive development for him nonetheless.

\textit{A Legendary (5)} success goes into the realm of heroic fantasy in many ways, influencing the character’s whole situation in a positive manner. 
New and even slightly unrelated opportunities might appear — meeting unexpected allies or friends waiting at the cliff top might reflect this dramatic influence, for example.

Characters making \textit{Ultimate (6)} Ability checks should not have to check the same Ability again in this session, barring dramatic circumstance or player initiative. 
It would be reasonable to decide that a Climbing success at this level overflows into long-term success, making the whole cliff-climbing a non-issue for the rest of the journey.

Finally, a \textit{Transcendent (7)} check, as the name implies, really means that the character broke the limits of what the Ability actually means.
Protagonist will leave the story but will leave their mark

\subsubsection{Bonus and penalty dice}
Challenge might have either bonus or penalty.
If roll has both, they are removed one-to-one bases until there is only one left.
When rolling challenge, in addition to three fudge dice, roll dice to amount of bonus or penalty.
After roll, choose three best die, if you were rolling with bonus dice, or three worst, if you were rolling with penalty dice.

Bonus die can be gained in several ways.
Common way is to spend one from Pool relating to the ability to gain single bonus die.
This can be only done once per check.

Characters can also assist each other with skills.
This is done with Challenge.
Amount of success is added as bonus die, however, failing can cause addition of penalty die or failure on whole task.
Remember, that before rolling, stakes need to be clear.
For any challenge, only one thing can be helping it, this can however be combined with bonus die which is gained by spending from related Pool.

Effects can be created by performing a challenge.
Result of effect is the effects strength.
Effects are delayed assist.
Effect could be, for example, "Reasonable argument given to mass"(Reason, 4).
That effect can be used, when it would make sense that thing to be of assistance.
Effect persist until it is used, or relating pool is refreshed.
Losing effect on refresh can be avoided by immediatly spending one from the pool.
Effect can also be lost if that happens in story (for example, you lose the forged documents).

\GM{} is allowed also to add circumstansial penalties.
\GM{} has option of either adding one or two penalty die, in order to communicate that the task has much hinderance.
No more should be added, instead it is better to say that task will result in failure always.

\subsubsection{Deciding to perform challenge}
As story moves during free play, situation of scene might at some point require challenge.
Any player might want to call for challenge, it is always up to discussion.
Ability is selected, and it is possible that without proper ability, something cannot be tried (If ability is missing because of oversight, it should be added, if it is rolled).
It is very meaningful decision to not have some common ability, so it should be ok to all to agree, that because of complete lack of skill, character is unable to even attempt something.
At this point, different angle might be suggested.
Sometimes, there are multiple abilities that can fit the spot, group should not be too picky on the matter.
If attempted task fits the description, it does not matter, if there is something that might fit better.

After decision is made on what ability should be used, stakes should be discussed.
Stakes define what will happen on success, and what will happen on failure.
It is important that player understands what kind of risk is their character taking.
This will make rolls significant.
Not all the results need to be defined, but price of failure should be clear, as well as what will be gained on success.

\subsubsection{Anatomy of challenge}
To collect these things together:
\begin{enumerate}
\item Recognize tense situation where Challenge is required
\item Select ability to use, and how it can overcome the challenge
\item Player and \gM discuss about stakes of the conflict
\item \GM{} can assign circumstansial penalties
\item Player can use existing effects, pool or some other abilities to gain bonus dice
\item Player makes decision, knowing the stakes, whether character will try to perform said action
\item Player rolls dice 3 + bonus/penalty die count
\item Player chooses three best/worst results from dice thrown
\item Result is narrated.
\end{enumerate}

There is also other type of challenge, that is not yet been discussed.
If someone with abilities is resisting the challenge, this challenge resolution is not directly used, instead it is conflict, and will be discussed in next section (Section \ref{ssec:conflict})

\pagebreak
