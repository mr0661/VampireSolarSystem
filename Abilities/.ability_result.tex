\newcommand{\abilityMonsterCombat}{
\begin{minipage}[b]{5cm}
\tiny{
\begin{center}
\begin{tabular}{ |p{5cm}| }
\hline
\rowcolor{black}\multicolumn{1}{|c|}{\nameCommand{Monster combat(V)}}
\\
\\[0pt]
This is used when fighting against monsters e.g. vampires. Whenever facing unnatural beast, that uses it supernatural powers those can be turned in to the weakness. It requires that you have prior knowledge of these tricks. It also requires that you have proper tools. These proper tools to counter monsters abilities can be produced with enought time and planning. This is only ability that allows battle against supernatural beings without penalty die. This is also what is used when two monsters fight against each other.
\\[0pt]
\\[0pt]
\hline
\end{tabular}
\end{center}
}
\end{minipage} %%\hspace{-10pt}
}
\newcommand{\abilityMonsterCombat}{
Monster combat (V)
}
\newcommand{\abilityUnarmedCombat}{
\begin{minipage}[b]{5cm}
\tiny{
\begin{center}
\begin{tabular}{ |p{5cm}| }
\hline
\rowcolor{black}\multicolumn{1}{|c|}{\nameCommand{Unarmed combat(V)}}
\\
\\[0pt]
This can only be used when your opponent does not have any weapon. It also requires that opponent fights like normal human being.
\\[0pt]
\\[0pt]
\hline
\end{tabular}
\end{center}
}
\end{minipage} %%\hspace{-10pt}
}
\newcommand{\abilityUnarmedCombat}{
Unarmed combat (V)
}
\newcommand{\abilityAnimalCombat}{
\begin{minipage}[b]{5cm}
\tiny{
\begin{center}
\begin{tabular}{ |p{5cm}| }
\hline
\rowcolor{black}\multicolumn{1}{|c|}{\nameCommand{Animal combat(V)}}
\\
\\[0pt]
This is used when fighting against animal. Animal needs to act like normal animal with normal animal properties. For example, this works against wolfs, horses, dogs, cat, and ravens. It can be used also against something that attacks like an animal. If fighting against something that has unnatural abilities but attacks like animal skill can be used with penalty.
\\[0pt]
\\[0pt]
\hline
\end{tabular}
\end{center}
}
\end{minipage} %%\hspace{-10pt}
}
\newcommand{\abilityAnimalCombat}{
Animal combat (V)
}
\newcommand{\abilityDuel}{
\begin{minipage}[b]{5cm}
\tiny{
\begin{center}
\begin{tabular}{ |p{5cm}| }
\hline
\rowcolor{black}\multicolumn{1}{|c|}{\nameCommand{Duel(V)}}
\\
\\[0pt]
Used when fighting in clear duel situation. This is used mainly to survive in such situations, not necessary to win it. If winning is only way to survive such event, then winning might be what success indicates. It requires that fight is clearly organized and follows some sort of rules. Duels are not typically to the death. This roll is typically contested by your opponent but is not always.
\\[0pt]
\\[0pt]
\hline
\end{tabular}
\end{center}
}
\end{minipage} %%\hspace{-10pt}
}
\newcommand{\abilityDuel}{
Duel (V)
}
\newcommand{\abilityRangedCombat}{
\begin{minipage}[b]{5cm}
\tiny{
\begin{center}
\begin{tabular}{ |p{5cm}| }
\hline
\rowcolor{black}\multicolumn{1}{|c|}{\nameCommand{Ranged combat(V)}}
\\
\\[0pt]
This is used when fighting against opponent that uses ranged weapon. Ability is opposed when fighting with proper weapons, e.g. ranged weapon or with shield. Some other tools or situation might also produce opposed roll. It can also be used while wielding melee weapon but then roll is likely to be parallel and with penalty. It is also possible to use this simply to avoid an attack.
\\[0pt]
\\[0pt]
\hline
\end{tabular}
\end{center}
}
\end{minipage} %%\hspace{-10pt}
}
\newcommand{\abilityRangedCombat}{
Ranged combat (V)
}
\newcommand{\abilityBattle}{
\begin{minipage}[b]{5cm}
\tiny{
\begin{center}
\begin{tabular}{ |p{5cm}| }
\hline
\rowcolor{black}\multicolumn{1}{|c|}{\nameCommand{Battle(V)}}
\\
\\[0pt]
Used when fighting with organized group. This is not for leading army, only participating in combat situation. This is used mainly to survive in such situations, not necessary to win it. In normal situations, this is not contested.
\\[0pt]
\\[0pt]
\hline
\end{tabular}
\end{center}
}
\end{minipage} %%\hspace{-10pt}
}
\newcommand{\abilityBattle}{
Battle (V)
}
\newcommand{\abilityBrawl}{
\begin{minipage}[b]{5cm}
\tiny{
\begin{center}
\begin{tabular}{ |p{5cm}| }
\hline
\rowcolor{black}\multicolumn{1}{|c|}{\nameCommand{Brawl(V)}}
\\
\\[0pt]
Used when fighting in situation that has large group fighting without coorination. This is used mainly to survive in such situations, not necessary to win it. In normal situations, this is not contested.
\\[0pt]
\\[0pt]
\hline
\end{tabular}
\end{center}
}
\end{minipage} %%\hspace{-10pt}
}
\newcommand{\abilityBrawl}{
Brawl (V)
}
\newcommand{\abilityMemory}{
\begin{minipage}[b]{5cm}
\tiny{
\begin{center}
\begin{tabular}{ |p{5cm}| }
\hline
\rowcolor{black}\multicolumn{1}{|c|}{\nameCommand{Memory(R)}}
\\
\\[0pt]
Memory is skill that many value. It is about recalling something. Used when it is not clear that someone could remember such detail. However, everything that is stated to happen is assumed to be in such importance, that characters remember them for rest of their lives. Memory can be used to bring interesting facts into the game. Player could ask that when new NPC is introduced to remember something about them. If they succeed, they remember, and such fact should be generated in a spot. For example, character might remember that same phrase was used by her nemesis and can now suspect that this character knows her nemesis.
\\[0pt]
\\[0pt]
\hline
\end{tabular}
\end{center}
}
\end{minipage} %%\hspace{-10pt}
}
\newcommand{\abilityMemory}{
Memory (R)
}
\newcommand{\abilitySports}{
\begin{minipage}[b]{5cm}
\tiny{
\begin{center}
\begin{tabular}{ |p{5cm}| }
\hline
\rowcolor{black}\multicolumn{1}{|c|}{\nameCommand{Sports(V)}}
\\
\\[0pt]
This used all manner of sport. It is about strenth possessed. Typically used in general athletic performances. This can be when swimming, sprinting, falling etc.
\\[0pt]
\\[0pt]
\hline
\end{tabular}
\end{center}
}
\end{minipage} %%\hspace{-10pt}
}
\newcommand{\abilitySports}{
Sports (V)
}
\newcommand{\abilitySpeak}{
\begin{minipage}[b]{5cm}
\tiny{
\begin{center}
\begin{tabular}{ |p{5cm}| }
\hline
\rowcolor{black}\multicolumn{1}{|c|}{\nameCommand{Speak(R)}}
\\
\\[0pt]
Speak is about explaining yourself. This can be used when you are arguing. Sometimes it takes a form of a inspiring speach. Speak is used when people are reasonable and are willing to listen for it. Witty banter can be in speak but is usually reserced for Charm and Deceit
\\[0pt]
\\[0pt]
\hline
\end{tabular}
\end{center}
}
\end{minipage} %%\hspace{-10pt}
}
\newcommand{\abilitySpeak}{
Speak (R)
}
\newcommand{\abilityBarter}{
\begin{minipage}[b]{5cm}
\tiny{
\begin{center}
\begin{tabular}{ |p{5cm}| }
\hline
\rowcolor{black}\multicolumn{1}{|c|}{\nameCommand{Barter(I)}}
\\
\\[0pt]
Bartering is something that every human has done. This is typically done when trying to buy something. It is both understanding value of something as it is to have someone to miss understand value. Though it is commonly used when purchasing goods, skilled barterer could convince bandits that he is more valuable alive than he is dead. Use this to acquire something.
\\[0pt]
\\[0pt]
\hline
\end{tabular}
\end{center}
}
\end{minipage} %%\hspace{-10pt}
}
\newcommand{\abilityBarter}{
Barter (I)
}
\newcommand{\abilityDeceit}{
\begin{minipage}[b]{5cm}
\tiny{
\begin{center}
\begin{tabular}{ |p{5cm}| }
\hline
\rowcolor{black}\multicolumn{1}{|c|}{\nameCommand{Deceit(I)}}
\\
\\[0pt]
Deceit is art of making people believe you. It is about creating impression that something is the truth. This archieved with many tricks, e.g. giving misleading information, lying, or telling the truth. Deceit does not necessary need to be about giving false information, though it is what most people use it. Sometimes it can be using weakness of mind against itself and help to person understand the reality. As the tricks are same, ability is the same.
\\[0pt]
\\[0pt]
\hline
\end{tabular}
\end{center}
}
\end{minipage} %%\hspace{-10pt}
}
\newcommand{\abilityDeceit}{
Deceit (I)
}
\newcommand{\abilityCharm}{
\begin{minipage}[b]{5cm}
\tiny{
\begin{center}
\begin{tabular}{ |p{5cm}| }
\hline
\rowcolor{black}\multicolumn{1}{|c|}{\nameCommand{Charm(I)}}
\\
\\[0pt]
Charm is about being liked. This means making good first impressions and eventually friends. It is also about manipulation with subtle social means. Romantic conquests are also done with charm.
\\[0pt]
\\[0pt]
\hline
\end{tabular}
\end{center}
}
\end{minipage} %%\hspace{-10pt}
}
\newcommand{\abilityCharm}{
Charm (I)
}