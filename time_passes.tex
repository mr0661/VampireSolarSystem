Common theme that people want to explore with immortality, is passing of time.
This chapter is dedicated to help to explore, when months, years, or centuries pass, while vampire characters remain mostly unchanged. % or stagnant?
This is called time skip, meaning that we move our focus to more interesting time.
Not every game needs to have the passing of time.
During time skip, characters might, for example, learn new skills, consolidate power, heal wounds, build friendships, or make enemies.

First step is to decide that spot is good for doing the time skip.
It might come naturally but normally it requires that everyone on the table knows that there is going to be a time skip.
Good spot is after something major or intense has been just resolved.
The time skip should happen when there is still things ongoing but not immediately at least for every character.
It is also totally acceptable that someone asks that should we start a time skip, and someone respons not yet. % TODO: example?

After deciding that it is good time for a time skip, next it should be decided where and when time skip will take a place.
You can be open discussion but it can also be more interesting to let players make this decision. % erottelu player storyteller?
One inspirational way to do it, is to have one player to make decision on time, 
	and other player makes decision on where characters meet.
When deciding time, player can give more details on time, general stuff that everyone in the world knows.
When deciding location, description of scene and reason why characters meet there should be given.

\textit{First player want that where time skip ends is in 1806 AD. They want to specify that this is date is important as inquisition is abolished.} % TODO: better example?
\textit{Second player want that play will continue in Rome. They tell, that vampires are gathering there, as vampire ruler of Rome is going to make important declaration.}

Now players know how much time is covered with the time skip.
It might be months, or it might be years.
Vampires do not need to just be passive during this time.
Players describe what their characters were doing during this time.
This does not need to be extremely detailed, details can be figured out later.
Only important things should be outlined during this time.
I would recommend just specifying one thing but this is not strict sometimes there are several important things.
These typically cause skill check for each player character. % TODO: skill check?
% TODO: add example

As years pass, characters can also progress.
This is represented with gaining new secrets (Section~\ref{ssec:secrets}).
Normal rules of acquiring secrets still apply.
Secret gaining:
\begin{itemize}
\item[Month] 0 secret
\item[Year] 2 secret
\item[Decade] 4 secret
\item[Century] 6 secret
\item[Millenia] 8 secret
\end{itemize}
If people are not certain what secrets to buy, these can also be left undecided, and chosen when they would come up in story.
When these are chosen after time skip, it would be good to hear story, where character learned this secret
   \textit{(e.g. When Isabella was in France hiding from wrath of the bishop, she learned to synergize her Vagrancy with other skills) }.
Main purpose of this rule is to not rush people to make bad decisions.
Some secrets might difficult to acquire, and should be done immediatetly, and might require explanation how those are gained.
   \textit{(e.g. Isabella wanted to learn how to read people minds. That is reason why she went to the bishop, only person she knew that would accept money as payment ) }.
Details should are decided on table.

