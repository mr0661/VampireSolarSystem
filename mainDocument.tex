\documentclass[a4paper, 12pt, finnish]{article}

\usepackage{bookman}
\usepackage{calc}
\usepackage[]{hyperref}
\usepackage[utf8]{inputenc}
\usepackage[table]{xcolor}
\usepackage{chngpage}
\usepackage{multicol}
\usepackage{enumitem}
\usepackage{xstring}

\usepackage[english]{babel}
\usepackage{blindtext}
\usepackage{catchfile}

\setlength{\parindent}{0em}
\setlength{\parskip}{1em}
\setlength{\footskip}{100pt}
\CatchFileDef{\HEAD}{.git/refs/heads/master}{}
\newcommand{\gitrevision}{%
  \StrLeft{\HEAD}{7}%
}

\newcommand{\comment}[1]{}

\begin{document}

\newcommand{\nameSpacerG}[0]{\color{black}\Large{g}}
\newcommand{\nameSpacer}[0]{\makebox[0pt][l]{\nameSpacerG}}

\newcommand{\nameSize}[1]{\color{white}\small{ \textbf{#1}}}
\newcommand{\nameCommand}[1]{\nameSpacer\nameSize{#1}}

\newcommand{\para}[1]{\paragraph{#1}\mbox{}\\}

\newcommand{\emptyTable}[0]{
\begin{minipage}[b]{5cm}
\begin{center}
\begin{tabular}{ p{5cm} } 
 \
\end{tabular}
\end{center}
\end{minipage} \hfill
}

\newcommand{\cols}[1]{
\begin{adjustwidth}{-1.25cm}{-1.25cm}
\begin{multicols} {3}
#1
\end{multicols}
\end{adjustwidth}
}
\newcommand{\thisBook}[0]{this book}
\newcommand{\ThisBook}[0]{This book}

\newcommand{\humanity}[0]{Moral}
\newcommand{\Humanity}[0]{Moral}
\newcommand{\humanities}[0]{Morals}
\newcommand{\Humanities}[0]{Morals}

\newcommand{\gM}[0]{Game master}
\newcommand{\GM}[0]{game master}

\newcommand{\campaign}[0]{campaign}
\newcommand{\Campaign}[0]{Campaign}
\newcommand{\session}[0]{session}
\newcommand{\Session}[0]{Session}

\newcommand{\abilityMonsterCombat}{
\begin{minipage}[b]{5cm}
\tiny{
\begin{center}
\begin{tabular}{ |p{5cm}| }
\hline
\rowcolor{black}\multicolumn{1}{|c|}{\nameCommand{Monster combat(V)}}
\\
\\[0pt]
This is used when fighting against monsters e.g. vampires. Whenever facing unnatural beast, that uses it supernatural powers those can be turned in to the weakness. It requires that you have prior knowledge of these tricks. It also requires that you have proper tools. These proper tools to counter monsters abilities can be produced with enought time and planning. This is only ability that allows battle against supernatural beings without penalty die. This is also what is used when two monsters fight against each other.
\\[0pt]
\\[0pt]
\hline
\end{tabular}
\end{center}
}
\end{minipage} %%\hspace{-10pt}
}
\newcommand{\abilityMonsterCombat}{
Monster combat (V)
}
\newcommand{\abilityUnarmedCombat}{
\begin{minipage}[b]{5cm}
\tiny{
\begin{center}
\begin{tabular}{ |p{5cm}| }
\hline
\rowcolor{black}\multicolumn{1}{|c|}{\nameCommand{Unarmed combat(V)}}
\\
\\[0pt]
This can only be used when your opponent does not have any weapon. It also requires that opponent fights like normal human being.
\\[0pt]
\\[0pt]
\hline
\end{tabular}
\end{center}
}
\end{minipage} %%\hspace{-10pt}
}
\newcommand{\abilityUnarmedCombat}{
Unarmed combat (V)
}
\newcommand{\abilityAnimalCombat}{
\begin{minipage}[b]{5cm}
\tiny{
\begin{center}
\begin{tabular}{ |p{5cm}| }
\hline
\rowcolor{black}\multicolumn{1}{|c|}{\nameCommand{Animal combat(V)}}
\\
\\[0pt]
This is used when fighting against animal. Animal needs to act like normal animal with normal animal properties. For example, this works against wolfs, horses, dogs, cat, and ravens. It can be used also against something that attacks like an animal. If fighting against something that has unnatural abilities but attacks like animal skill can be used with penalty.
\\[0pt]
\\[0pt]
\hline
\end{tabular}
\end{center}
}
\end{minipage} %%\hspace{-10pt}
}
\newcommand{\abilityAnimalCombat}{
Animal combat (V)
}
\newcommand{\abilityDuel}{
\begin{minipage}[b]{5cm}
\tiny{
\begin{center}
\begin{tabular}{ |p{5cm}| }
\hline
\rowcolor{black}\multicolumn{1}{|c|}{\nameCommand{Duel(V)}}
\\
\\[0pt]
Used when fighting in clear duel situation. This is used mainly to survive in such situations, not necessary to win it. If winning is only way to survive such event, then winning might be what success indicates. It requires that fight is clearly organized and follows some sort of rules. Duels are not typically to the death. This roll is typically contested by your opponent but is not always.
\\[0pt]
\\[0pt]
\hline
\end{tabular}
\end{center}
}
\end{minipage} %%\hspace{-10pt}
}
\newcommand{\abilityDuel}{
Duel (V)
}
\newcommand{\abilityRangedCombat}{
\begin{minipage}[b]{5cm}
\tiny{
\begin{center}
\begin{tabular}{ |p{5cm}| }
\hline
\rowcolor{black}\multicolumn{1}{|c|}{\nameCommand{Ranged combat(V)}}
\\
\\[0pt]
This is used when fighting against opponent that uses ranged weapon. Ability is opposed when fighting with proper weapons, e.g. ranged weapon or with shield. Some other tools or situation might also produce opposed roll. It can also be used while wielding melee weapon but then roll is likely to be parallel and with penalty. It is also possible to use this simply to avoid an attack.
\\[0pt]
\\[0pt]
\hline
\end{tabular}
\end{center}
}
\end{minipage} %%\hspace{-10pt}
}
\newcommand{\abilityRangedCombat}{
Ranged combat (V)
}
\newcommand{\abilityBattle}{
\begin{minipage}[b]{5cm}
\tiny{
\begin{center}
\begin{tabular}{ |p{5cm}| }
\hline
\rowcolor{black}\multicolumn{1}{|c|}{\nameCommand{Battle(V)}}
\\
\\[0pt]
Used when fighting with organized group. This is not for leading army, only participating in combat situation. This is used mainly to survive in such situations, not necessary to win it. In normal situations, this is not contested.
\\[0pt]
\\[0pt]
\hline
\end{tabular}
\end{center}
}
\end{minipage} %%\hspace{-10pt}
}
\newcommand{\abilityBattle}{
Battle (V)
}
\newcommand{\abilityBrawl}{
\begin{minipage}[b]{5cm}
\tiny{
\begin{center}
\begin{tabular}{ |p{5cm}| }
\hline
\rowcolor{black}\multicolumn{1}{|c|}{\nameCommand{Brawl(V)}}
\\
\\[0pt]
Used when fighting in situation that has large group fighting without coorination. This is used mainly to survive in such situations, not necessary to win it. In normal situations, this is not contested.
\\[0pt]
\\[0pt]
\hline
\end{tabular}
\end{center}
}
\end{minipage} %%\hspace{-10pt}
}
\newcommand{\abilityBrawl}{
Brawl (V)
}
\newcommand{\abilityMemory}{
\begin{minipage}[b]{5cm}
\tiny{
\begin{center}
\begin{tabular}{ |p{5cm}| }
\hline
\rowcolor{black}\multicolumn{1}{|c|}{\nameCommand{Memory(R)}}
\\
\\[0pt]
Memory is skill that many value. It is about recalling something. Used when it is not clear that someone could remember such detail. However, everything that is stated to happen is assumed to be in such importance, that characters remember them for rest of their lives. Memory can be used to bring interesting facts into the game. Player could ask that when new NPC is introduced to remember something about them. If they succeed, they remember, and such fact should be generated in a spot. For example, character might remember that same phrase was used by her nemesis and can now suspect that this character knows her nemesis.
\\[0pt]
\\[0pt]
\hline
\end{tabular}
\end{center}
}
\end{minipage} %%\hspace{-10pt}
}
\newcommand{\abilityMemory}{
Memory (R)
}
\newcommand{\abilitySports}{
\begin{minipage}[b]{5cm}
\tiny{
\begin{center}
\begin{tabular}{ |p{5cm}| }
\hline
\rowcolor{black}\multicolumn{1}{|c|}{\nameCommand{Sports(V)}}
\\
\\[0pt]
This used all manner of sport. It is about strenth possessed. Typically used in general athletic performances. This can be when swimming, sprinting, falling etc.
\\[0pt]
\\[0pt]
\hline
\end{tabular}
\end{center}
}
\end{minipage} %%\hspace{-10pt}
}
\newcommand{\abilitySports}{
Sports (V)
}
\newcommand{\abilitySpeak}{
\begin{minipage}[b]{5cm}
\tiny{
\begin{center}
\begin{tabular}{ |p{5cm}| }
\hline
\rowcolor{black}\multicolumn{1}{|c|}{\nameCommand{Speak(R)}}
\\
\\[0pt]
Speak is about explaining yourself. This can be used when you are arguing. Sometimes it takes a form of a inspiring speach. Speak is used when people are reasonable and are willing to listen for it. Witty banter can be in speak but is usually reserced for Charm and Deceit
\\[0pt]
\\[0pt]
\hline
\end{tabular}
\end{center}
}
\end{minipage} %%\hspace{-10pt}
}
\newcommand{\abilitySpeak}{
Speak (R)
}
\newcommand{\abilityBarter}{
\begin{minipage}[b]{5cm}
\tiny{
\begin{center}
\begin{tabular}{ |p{5cm}| }
\hline
\rowcolor{black}\multicolumn{1}{|c|}{\nameCommand{Barter(I)}}
\\
\\[0pt]
Bartering is something that every human has done. This is typically done when trying to buy something. It is both understanding value of something as it is to have someone to miss understand value. Though it is commonly used when purchasing goods, skilled barterer could convince bandits that he is more valuable alive than he is dead. Use this to acquire something.
\\[0pt]
\\[0pt]
\hline
\end{tabular}
\end{center}
}
\end{minipage} %%\hspace{-10pt}
}
\newcommand{\abilityBarter}{
Barter (I)
}
\newcommand{\abilityDeceit}{
\begin{minipage}[b]{5cm}
\tiny{
\begin{center}
\begin{tabular}{ |p{5cm}| }
\hline
\rowcolor{black}\multicolumn{1}{|c|}{\nameCommand{Deceit(I)}}
\\
\\[0pt]
Deceit is art of making people believe you. It is about creating impression that something is the truth. This archieved with many tricks, e.g. giving misleading information, lying, or telling the truth. Deceit does not necessary need to be about giving false information, though it is what most people use it. Sometimes it can be using weakness of mind against itself and help to person understand the reality. As the tricks are same, ability is the same.
\\[0pt]
\\[0pt]
\hline
\end{tabular}
\end{center}
}
\end{minipage} %%\hspace{-10pt}
}
\newcommand{\abilityDeceit}{
Deceit (I)
}
\newcommand{\abilityCharm}{
\begin{minipage}[b]{5cm}
\tiny{
\begin{center}
\begin{tabular}{ |p{5cm}| }
\hline
\rowcolor{black}\multicolumn{1}{|c|}{\nameCommand{Charm(I)}}
\\
\\[0pt]
Charm is about being liked. This means making good first impressions and eventually friends. It is also about manipulation with subtle social means. Romantic conquests are also done with charm.
\\[0pt]
\\[0pt]
\hline
\end{tabular}
\end{center}
}
\end{minipage} %%\hspace{-10pt}
}
\newcommand{\abilityCharm}{
Charm (I)
}
\newcommand{\AllKeyName}{
\KeyNameExampleKeyTwo{}
\KeyNameExampleKey{}
}
\newcommand{\AllKey}{
\KeyExampleKeyTwo{}
\KeyExampleKey{}
}
\newcommand{\KeyExampleKeyTwo}{
\begin{minipage}[b]{5cm}
\tiny{
\begin{center}
\begin{tabular}{ |p{5cm}| }
\hline
\rowcolor{black}\multicolumn{1}{|c|}{\nameCommand{My key 2}}
\\
\\[0pt]
Core text starts here Core text starts here Core text starts here
\vspace{-1mm}
\begin{itemize}
\setlength\itemsep{-2pt}
\item[1xp:] Something minor
%\item[2xp:] :xp2:
\item[3xp:] Something important Something important Something important Something important Something important
%\item[5xp:] :xp5:
\item[buyoff:] Some twist
\end{itemize}
\\
%\multicolumn{1}{|r p{4cm}|}{xp1: & cp1} %%& \multicolumn{1}{p{4cm}|}{Something minor} \\
%%\multicolumn{1}{|r}{xp2:} & \multicolumn{1}{l|}{:xp2:} \\
%\multicolumn{1}{|r}{xp3:} & \multicolumn{1}{l|}{Something important Something important Something important Something important Something important} \\
%%\multicolumn{1}{|r}{xp5:} & \multicolumn{1}{l|}{:xp5:} \\
%\multicolumn{1}{|r}{buyoff:} & \multicolumn{1}{l|}{Some twist} \\
\hline
\end{tabular}
\end{center}
}
\end{minipage}
}
\newcommand{\KeyNameExampleKeyTwo}{
My key 2
}
\newcommand{\KeyExampleKey}{
\begin{minipage}[b]{5cm}
\tiny{
\begin{center}
\begin{tabular}{ |p{5cm}| }
\hline
\rowcolor{black}\multicolumn{1}{|c|}{\nameCommand{My key}}
\\
\\[0pt]
Core text starts here Core text starts here Core text starts here
\vspace{-1mm}
\begin{itemize}
\setlength\itemsep{-2pt}
\item[1xp:] Something minor
\item[2xp:] Something larger
%\item[3xp:] :xp3:
\item[5xp:] Something major
\item[buyoff:] Some twist
\end{itemize}
\\
%\multicolumn{1}{|r p{4cm}|}{xp1: & cp1} %%& \multicolumn{1}{p{4cm}|}{Something minor} \\
%\multicolumn{1}{|r}{xp2:} & \multicolumn{1}{l|}{Something larger} \\
%%\multicolumn{1}{|r}{xp3:} & \multicolumn{1}{l|}{:xp3:} \\
%\multicolumn{1}{|r}{xp5:} & \multicolumn{1}{l|}{Something major} \\
%\multicolumn{1}{|r}{buyoff:} & \multicolumn{1}{l|}{Some twist} \\
\hline
\end{tabular}
\end{center}
}
\end{minipage}
}
\newcommand{\KeyNameExampleKey}{
My key
}
\newcommand{\AllSecrets}{
\secretmyTopic{}
\secretConditioning{}
}
\newcommand{\secretmyTopic}{
\tiny{
\begin{center}
\begin{tabular}{ |p{5cm}| }
\hline
\rowcolor{black}
\multicolumn{1}{|c|}{\nameCommand{something}}
\\
\\[0pt]
Core text starts here Core text starts here Core text starts here
 \textbf{Requirements:} own other secret
% \textbf{Requirements:} :requirements:
 \textbf{Cost:} 1 from any pool
\\[0pt]
\\[0pt]
\hline

\end{tabular}
\end{center}
}
}
\newcommand{\secretConditioning}{
\tiny{
\begin{center}
\begin{tabular}{ |p{5cm}| }
\hline
\rowcolor{black}
\multicolumn{1}{|c|}{\nameCommand{Secret of conditioning}}
\\
\\[0pt]
You are specifically conditioned to use one of your pools when possible. Whenever you spend from specified pool add one additianal bonus die.
% \textbf{Requirements:} :req:
% \textbf{Requirements:} :requirements:
 \textbf{Cost:} 1 extra from chosen pool
\\[0pt]
\\[0pt]
\hline

\end{tabular}
\end{center}
}
}


\title{\ThisBook}
\author{Ossi Noita}
\date{\today{} (\gitrevision)}

\comment{
%% ABILITIES %%

\textbf{Passive abilities:}

\textbf{Passive abilities} are universal to every human and can only be utilized as defensive.
However, almost any time person wants to be defensive, they can choose to use passive ability.

\cols{
\PassiveAbility{}
}

\textbf{Universal abilities:}

\textbf{Universal abilities} are universal to humans in any time or place.

\cols{
\UniversalAbility{}
}

\pagebreak

\textbf{Combat abilities:}

\textbf{Combat abilities} can be are used to defeat opponent with sheer force. In general sense, ability describes how well can you fight against an opponent of specific type or survival in situations. Some combat abilities are general while other are universal. Abilities with combat in name are to defeat opponent of that kind and others are survival in combat situations.


\cols{
\CombatAbility{}
}

\pagebreak

\textbf{General abilities:}

\textbf{General abilities} can be acquire during any time period but might be influenced by culture. These abilties are not possed by every one. General abilities are: \GeneralAbilityName{}

\cols{
\GeneralAbility{}
}

\pagebreak

\textbf{Era abilities:}

\textbf{Era abilities} are useful and acquirable only in certain eras. These are mostly just examples, as there can be quite a lot of these: \EraAbilityName{}

\cols{
\EraAbility{}
}

\pagebreak

\textbf{Vampiric abilities:}

\textbf{Vampiric abilities} are useful and acquirable only in certain eras. These are mostly just examples, as there can be quite a lot of these: \VampiricAbilityName{}

\cols{
\VampiricAbility{}
}

\pagebreak

%% SECRETS %%

\textbf{Universal secrets:}

\textbf{Universal secrets} general secrets that can be considerd to bring some specific aspect of character.

\cols{
\UniversalSecret{}
}

\textbf{Status secrets:}

\textbf{Status secrets} are secrets that nomite some extremely important status. 
Person with those attributes do not need to have mathcing secrets if they do not expect to gain any banafits form them.
Having relevant status secret might give bonus die on some situations or automatic success on others.
Sometimes you might need to have secret to even try.
It is not necessary to have several status secrets if all apply, just choose most describing then (Wealthy king would might be Nobility where rich baron might be Wealthy).
However, if these are unrelated to each other, for example character is vampiric nobility and french political leader, it might make sense to have both.

\cols{
\StatusSecret{}
}


\textbf{Culture secrets:}

\textbf{Culture secrets} relate to specific culture.
To learn these you typically need teacher.
These are needed to not seem foreing in those cultures.
In general sense, languages are acquired through this but some basic knowledge on these can be acquired without need to buy secrets.
For example, it might be typical that people in some culture know several languages.
Then person with that background would also know several languages.
In almost every situation, two person who meet are able to communicate with each other.
In addition, two people from foreign culture are able to communicate with each other without outsider understanding (though there might be exeptions).
With these secret, you do not need to use \AbilityNameDeceit{} on people so that they would believe you to be one of them.

\cols{
\CultureSecret{}
}

\textbf{Ability related secrets:}

\textbf{Ability related secrets} augment uses of some abilities. 
These do not perform miracles instead allow user to utize their abilities.
Many of these secrets will take time and perhaps money to use.

\cols{
\AbilitySecret{}
}


\pagebreak

\textbf{Vampiric secrets:}

\textbf{Vampiric secrets} represent powers vampires posses.

\textbf{Vampiric strengths } is vigor based ability. 
Typically secrets are used in combat but might have uses also elsewhere.
If used in combat, vampire is then considered as a Monster.


\cols{
\VampiricStrengthSecret{}
}

\textbf{Vampiric mysticism } is instinct based ability.
These secret provide access to some reality shaping abilities vampires can posses.
These are strong and difficult to use.


\cols{
\VampiricMysticismSecret{}
}

\textbf{Vampiric mind control }

\cols{
\VampiricMindControlSecret{}
}

\pagebreak

%% KEYS %%

\textbf{Key}

Most important and difficult key is key of humanity. It consist of several steps where you can lose humanity slowly. 
Each vampire has set of principles.
More you have, more easily you can function in society.
Maximum number is 7 and minimum is 1.
If all principles are lost, vampire loses its mind.
As more and more principles are lost vampire becomes more obsessive on the remaining ones.
It is possible to develop other principles but whenever principle is lost it cannot be replaced.


% OTA uusi "vampiric curse"
\begin{itemize}
\setlength\itemsep{-2pt}
\item[7] normal vampiric curses still apply
\item[6-] it costs single point from any pool to wake up. If you do can't, you slumber
\item[5-] it costs 1 Vigor to appear as a living, to have bodily functions as a living
\item[4-] you acquire some mental illness
\item[3-] you are unable to eat regular food or drink
\item[2-] you cannot consume animal blood
\item[1] you cannot spend Vigor anymore to appear as a human.
\end{itemize}

% kerrallaan saa olla enintään kolme, toimivat kuin avaimet, mutta ilmaisia
Example principles are:
\begin{itemize}
\setlength\itemsep{-2pt}
\item Kindness towards others
\item Do onto others as you would have them to do to you
\item Justice and fairness
\item Right of everyone to decide their own fate
\item Right for everyone to their own bodies
\item Respect for human life
\item Integrity of your own body
\end{itemize}

You can write your own. These can also be changed between sessions. However, if one is lost it cannot be replaced. It will be marked as broken principle and it is permanent loss.

Key of Humanity is linked with these. As these describe principles, Key of Humanity is how these can be lost.

\KeyHumanity{}

\KeyNameHumanity{} speaks of remorse. If you break a principle you can:
\begin{itemize}
\item[a.] Buyoff principle. Cross over one principle that you broke with an action, take 10 experience. Vampire will always have \KeyNameHumanity{}
\item[b.] Feel remorse. Take Reason damage for each principle you broke. Each broken principle can contribute between 1 - 3 points of damage. If damage would be larger than largest damage track remorse is impossible.
\end{itemize}

If principles are changed it means that the new thing is actually the real principle and has always been, other has been self deception.
If vampire has broken new principle he has to feel remorse from each of those missteps.
Braking principle should be informed decision from players.
It should not be clear accident.
However, it might be something that player did not want to happen.

\textit{For example, new vampire is unable to control his bloodlust. He therfore tries to drink his beloved forcefully. This is does not respect right that everyone has to their own bodies or respect for life. However as this would be unintentional, it is deemed that first contributes 2 point and second only 1. This means that if this happens, it either cause buyoff or remorse worth of 3 points.
Player has no ability to fight against this kind of bloodlust knowing what the stake (in addition of the beloved).}

\pagebreak
\textbf{Vampiric keys}
These are optional keys but usually required. Without any following penalties apply:
\begin{itemize}
\setlength\itemsep{-2pt}
\item You have natural instinct for blood. Whenever you see blood and do not feed some, take one instinct damage.
\item Existance feels to be without meaning. It costs one from Instinct pool and Reason pool to rise on morning (in addition to any other costs)
\end{itemize}
These are both discourage and to relate what happens if these urges are not directed elsewhere.
These keys shown are just some possible keys.
Important part is that they relate to vampiric existance.

\cols{
\VampiricKey{}
}

%% POOLS %%
\pagebreak
Pools are normal, expect for refershing condition:

\begin{itemize}
\item[\textbf{Vigor}] Drink human blood directly from vein or drink animal blood and do some physical enjoyment. This can be hard partying, doing drugs, sparring or anything that involves someone else.
\item[\textbf{Instinct}] Hunt for blood in social setting or interact with your kind in social setting.
\item[\textbf{Reason}] Try to understand your existence. Alternatively, drink blood in intellectually stimulating interaction such as poetry recital, a late-night philosophical debate or a chess match.
\end{itemize}
}

\begin{titlepage}
\maketitle
\end{titlepage}


\pagebreak
\tableofcontents
\pagebreak

\section{Terms}\label{sec:terms}
\section{Introduction}\label{sec:introduction}
	\subsection{About this book}\label{ssec:about}
		\ThisBook{} provides stand alone rules for playing tale and drama with vampires.
Protagonists might be vampires or some folk mingled with vampires somehow.
\ThisBook{} contains what you need to play game with vampires in it, where story is in focus.

\ThisBook{} uses Solar System for base of its rules.
Solar System (Eero Tuovinen 2008) is based on The Shadow of Yesterday (Clinton R. Nixon 2005).
However, inspiration to \thisBook{} is The World of Near (Eero Tuovinen 2009) which can be described as Solar System meeting The Shadow of Yesterday.
Solar System is rule set without setting, and purpose of \thisBook{}, is to provide vampire setting for these rules.
\ThisBook{} is designed so, that Solar System is not required reading, \thisBook{} contains all rules needed of play. % TODO change "this book"
Solar System is not retold here though, plan is to provide good solid rule set so people can play great, dramatic game with vampires.

% TODO Add blood pun here to say "Have good game"
\pagebreak

	\subsection{Roleplaying}\label{ssec:roleplaying}
		This section should explain what roleplaying is.
% TODO write this chapter
\pagebreak

	\subsection{Vampires}\label{ssec:vampires}
		This would introduce concept of vampires, and perhaps some terms.
\pagebreak

\section{System}\label{sec:System}
	\subsection{Overview}\label{ssec:overview}
		For system, we have a few concepts which are be good to know.
This chapter gives short overview.

\para{\nameref{ssec:keys}}
Keys are most important part of the system.
These tell what drives the protagonists.
Keys are also only way to advance protagonist capabilities.

\para{\nameref{ssec:abilities}}
Abilities describe how easily characters can have outcomes they desire.
These are skills, hard earned or innate.
These descibe what character can do, in order to have an affect to the world.

\para{\nameref{ssec:pools}}
These are personal resource of the character.
Abilities relate to some pool.
A pool can be used to improve changes of success when using ability related to that pool.
Pool dwindles as it is being used and can be restored to full by doing something that would refreshing related to that pool.
Differenct characters can have different pools or ways to refresh.

\para{\nameref{ssec:challenges}}
Sometimes situation arises to use ability.
This is called a challenge.
It might test character ability to perform surgery, or forge a sword, for example.

\para{\nameref{ssec:conflict}}
Conflict is two or more characters trying to achieve opposite outcomes utilizing their abilities.
Stakes of the conflict are desided by players of the characters, before determining what result actually takes the place.
It is also possible, that something unexpected disrupts the conflict.
All protagonists have also possibility of extending conflicts where they are, if they end up on losing side.

\para{\nameref{ssec:secrets}}
Each character can also have special abilities.
These can make success with abilities easier, or enable use of some mystical talents.
Characters need to earn their secrets.

\pagebreak


	\subsection{Keys}\label{ssec:keys}
		Keys represent goals for the main characters.
These are main feature of Solar System and most important part of character.
During game, these keys will change, defining important moments for characters.
Keys are lost (called buyoff) when the lover dies, grudge is forgiven, or vice is abondoned, for example.

Keys are mechanism of progression in the game for player characters.
Significance of character growth and presence in the story is represented experience characters gain.
For each five experience gained, character gains one new advantage, for them to spend as their player pleases.

Possible things that can be gained with advantage:
\begin{itemize}
\item[\textbf{cost}]
\item[1] Add new Key (this Section)
\item[1] Increase Ability from mediocre (Section \ref{ssec:abilities})
\item[2] Increase Ability from competent (Section \ref{ssec:abilities})
\item[3] Increase Ability from expert (Section \ref{ssec:abilities})
\item[1] Increase Pool before pool of 10 (Section \ref{ssec:pools})
\item[2] Increase Pool after pool of 10 (Section \ref{ssec:pools})
\item[1] Learn new Secret (Section \ref{ssec:secrets})
\end{itemize}

\pagebreak

	\subsection{Abilities}\label{ssec:abilities}
		Abilities define how character can affect world and story around them.
Greater the ability easier it is to use that particular skill to accomplish goals of the character.
Ability score typically represents skill but it can also be luck, or circumstances.

Someone who is not adequate might not even have ability score.
In those cases, character is not able to use that ability.
However, as story goes on, character might acquire new abilities or improve olds.
Acquiring new ability is free, though it means clear shift is character.
Improving ability cost advantages (Section \ref{ssec:keys}), equal to level which it is improved.

Each ability has one pool that relates to it (Section \ref{ssec:pools}).
Pool relating to ability can be utilized to get more out of that ability it is used (Section \ref{ssec:abilities}).

Anatomy of abilities is similar to each other but it is useful to categorize these.
Some categories are not exclusive; same ability might belong into multiple categories.
Abilities are not meant to be exclusive, sometimes it is up to player to choose what ability character uses in certain situation (Section \ref{ssec:abilities}).

\subsubsection{Passive abilities}\label{sssec:passive_abilities}
\hyperref[examples_passive_abilities]{Passive} abilites are special kind of abilities, that every character will have.
These are intrinsic to any sentient being and are last line of defense against threats.
In any situation, passive ability can be used, but it will only help to preserve the character; passive abilites are used only defensively.

\subsubsection{Universal abilities}\label{sssec:universal_abilities}
Some abilities are \hyperref[examples_universal_abilities]{universal} to everyone throughout time.
Every player character has had chance to learn any of these.
Some individuals did not learn these, but still posses basic understanding to acquire these when needed.

\subsubsection{General abilities}\label{sssec:general_abilities}
There are \hyperref[examples_general_abilities]{general} abilites that are common amongst humans, and have existed in humans throughout the history.
However, these are more professional abilities, things that required training to even have access to.
Almost everyone have some of these but having many is quite rare.

\subsubsection{Cultural abilities}\label{sssec:era_abilities}
Some abilites relate to time and place, \textit{i.e.} \hyperref[examples_era_abilities]{cultural}.
These might be easy or difficult to acquire but can easily be out of place to some.
When dealing with centuries old beings, they might posses abilities that are now unheard of and lack some common abilities even in basic level.
Some of the abilities need to be specified, for example \textbf{\AbilityNameReligion} requires to specify what religion it relates to.

\subsubsection{Combat abilities}\label{sssec:combat_abilities}
\hyperref[examples_combat_abilities]{Combat} abilities can be are used to defeat opponent with sheer force.
In general sense, ability describes how well can you fight against an opponent of specific type or survival in situations.
Some combat abilities are general while other are universal.
Abilities with combat in name are to handle opponent of that kind and others are survival in combat situations.

\pagebreak
\subsubsection{Example abilities}
\cols{
\paragraph{\nameref{sssec:passive_abilities}}\label{examples_passive_abilities}

\PassiveAbility

\paragraph{\nameref{sssec:universal_abilities}}\label{examples_universal_abilities}

\UniversalAbility

\paragraph{\nameref{sssec:general_abilities}}\label{examples_general_abilities}

\GeneralAbility

\paragraph{\nameref{sssec:era_abilities}}\label{examples_era_abilities}

\EraAbility

\paragraph{\nameref{sssec:combat_abilities}}\label{examples_combat_abilities}

\CombatAbility
}
\pagebreak

	\subsection{Pools}\label{ssec:pools}
		Every character have Pools which are reserves of strengths of characters.
Players may choose to have their characters spend their pool to boost their attempt while using abilities (Section \ref{ssec:abilities}).
As characters spend their pool, they become more exhausted, physically or mentally.
Eventually characters need to rest, in order to recover their pools.
Thus pools have two numbers, maximum and current.

Every character have these three pools and those represent different resources characters have:
\begin{itemize}
\item{\textbf{Vigor: }Capacity of exerting oneself in physical activity. Someone with low Vigor gets exhausted easily, and will not be able to really commit themselves to this physical activity.}
\item{\textbf{Instinct: }How well one can handle themselves during quick thinking. Someone with low Instinct might be somewhat somewhat slow, and might not be interested on social activities.}
\item{\textbf{Reason: }Shows how well character can push themselves when encountering mental challenges. Someone with low Reason can easily be duped, and is not willing to push themselves when thinking.}
\end{itemize}

Pools also relate to harm character might receive.
Pools are used to heal harm gained and each harm assossiates to one of the pools character has.
Harm is discussed more in (Section \ref{ssec:conflict}).

In addition of healing harm, pools are used to imporve changes on success when using abilities.
This happens when character is determined on compliting the task, and is ready to push themselves little more.

Last use for the pools is activating certain special abilities called secrets (Section \ref{ssec:secrets}).
Using some powers does not come cheaply, and utilizing those will consume something from character.

Pools are used to gain benefits mentioned above.
Pools can also be refreshed.
Characters can refresh their pools by relaxing and letting their guard down.
This also gives story some slow point to see what characters do to relax.
It is also opportunity to something important to happen.

Each pool is refreshed with different condition.
This condition also depends on what the character actually is.
Vampires relax inherently in different way than mortal humans.
Ways each character relaxes is discussed more on character section.
Whenever player wants to protagonist to refresh one or more of their pools, they can describe what protagonist will do in order to relax.
Each pool which condition is met by the way of relaxation is refreshed to full.

\pagebreak

	\subsection{Challenges}\label{ssec:challenges} % Tämä on perus tilanteisiin joissa heitetään noppaa
		Challenge is core resolution mechanism.
It is used to determine whether action relevant to the story succeeds or fails.
There is two types of challenges, simpler case is uncontested and other one is conflict (Section \ref{ssec:conflict}).
This chapter focuses on uncontested and any challenge in general.

\subsubsection{Base die}

Game uses fudge die.
These are six sided dice with three different sides: '-', ' ', and '+'.
Easy way to use any die as fudge die is to say that minus is 1 and 2, empty is 3 and 4, and plus is 5 an 6.

When story moves into situation, where success of character is questioned, challenge is called.
\GM{} and protagonists player should first discuss on stakes on the roll.
This is somewhat atypical for RPGs, it should be clear what are stakes when any roll is made.
This is decided by freely discussing what the attempt means in fiction.
It is up to the player to choose how much is resolved by any single roll, but stakes should match.
\GM{} is allowed to determine that if the protagonist would try it, it would automatically fail.

When challenge is called, player of the character throws three dice.
Player selects proper ability to use for the roll.
That ability score serves base result.
That is modified by total of the roll, each '-' decreases number by one, and each '+' increases number by one (empty do not have an effect).
Total value determines the result.
This can be intepreted in following way:
\begin{center}
\begin{tabular}{ c c l }
 \textbf{Total} & \textbf{Result} & \textbf{Description}\\ 
 0- & Failure & Character fails and will face consequences \\
 1 & Marginal & Success without any glamour \\
 2 & Good & Character is showing that they are good at what they do \\
 3 & Great & Something to be proud of \\
 4 & Amazing & Witnesses are impressed \\
 5 & Legendary & This act will inspire stories \\
 6 & Ultimate & Perfect execution, nobody can see any flaw in the act \\
 7 & Trascendent & World is permanently altered
\end{tabular}
\end{center}
% TODO Next part is copied
\textit{A Failure (0)} result does not have to mean that the character’s efforts just fizzle; 
	most of the time failure in the Solar System should be a cinematic affair full of dramatic consequence. 
A character who fails in a Climbing (V) check isn’t going to just “not get on the top of the cliff”, 
	he’s going to fall down from high and break his collarbone, being stranded alone and wounded in the wilderness.

\textit{A Marginal (1)} success might sound dull and predictable, but what it actually means is that the Story Guide is welcome to add a little twist, some complication to the success. 
The character might indeed get to the top of the cliff, but it might take so long that night falls during the climb, for example.

What characters really want is a \textit{Good (2)}, solid success, as that’s something to be proud of. 
Colleagues won’t find flaws in it, it’s the real thing. 
A bit dull, perhaps, but secure, as nobody’s going to interpret it as anything but solid fulfillment of the stakes.

\textit{A Great (3)} success is essentially master-level, it’s good and a little bit more, yet. 
The player would be well within his rights to describe how his character not only succeeds, but also does it in style. 
Getting to the top of the cliff is a given, but the character might as well find an easy route up, one that makes a second climb nigh trivial for him later on.

\textit{An Amazing (4)} success, as the chart says, is something that onlookers would be astounded by. 
A character making that kind of check deserves to have a solid edge further in the scene
	 — finding a sheltered cave in the cliffside to spend the night could be a positive twist supplied by the Story Guide, for example. 
Not exactly what the character tried to do, but a further positive development for him nonetheless.

\textit{A Legendary (5)} success goes into the realm of heroic fantasy in many ways, influencing the character’s whole situation in a positive manner. 
New and even slightly unrelated opportunities might appear — meeting unexpected allies or friends waiting at the cliff top might reflect this dramatic influence, for example.

Characters making \textit{Ultimate (6)} Ability checks should not have to check the same Ability again in this session, barring dramatic circumstance or player initiative. 
It would be reasonable to decide that a Climbing success at this level overflows into long-term success, making the whole cliff-climbing a non-issue for the rest of the journey.

Finally, a \textit{Transcendent (7)} check, as the name implies, really means that the character broke the limits of what the Ability actually means.
Protagonist will leave the story but will leave their mark

\subsubsection{Bonus and penalty dice}
Challenge might have either bonus or penalty.
If roll has both, they are removed one-to-one bases until there is only one left.
When rolling challenge, in addition to three fudge dice, roll dice to amount of bonus or penalty.
After roll, choose three best die, if you were rolling with bonus dice, or three worst, if you were rolling with penalty dice.

Bonus die can be gained in several ways.
Common way is to spend one from Pool relating to the ability to gain single bonus die.
This can be only done once per check.

Characters can also assist each other with skills.
This is done with Challenge.
Amount of success is added as bonus die, however, failing can cause addition of penalty die or failure on whole task.
Remember, that before rolling, stakes need to be clear.
For any challenge, only one thing can be helping it, this can however be combined with bonus die which is gained by spending from related Pool.

Effects can be created by performing a challenge.
Result of effect is the effects strength.
Effects are delayed assist.
Effect could be, for example, "Reasonable argument given to mass"(Reason, 4).
That effect can be used, when it would make sense that thing to be of assistance.
Effect persist until it is used, or relating pool is refreshed.
Losing effect on refresh can be avoided by immediatly spending one from the pool.
Effect can also be lost if that happens in story (for example, you lose the forged documents).

\GM{} is allowed also to add circumstansial penalties.
\GM{} has option of either adding one or two penalty die, in order to communicate that the task has much hinderance.
No more should be added, instead it is better to say that task will result in failure always.

\subsubsection{Deciding to perform challenge}
As story moves during free play, situation of scene might at some point require challenge.
Any player might want to call for challenge, it is always up to discussion.
Ability is selected, and it is possible that without proper ability, something cannot be tried (If ability is missing because of oversight, it should be added, if it is rolled).
It is very meaningful decision to not have some common ability, so it should be ok to all to agree, that because of complete lack of skill, character is unable to even attempt something.
At this point, different angle might be suggested.
Sometimes, there are multiple abilities that can fit the spot, group should not be too picky on the matter.
If attempted task fits the description, it does not matter, if there is something that might fit better.

After decision is made on what ability should be used, stakes should be discussed.
Stakes define what will happen on success, and what will happen on failure.
It is important that player understands what kind of risk is their character taking.
This will make rolls significant.
Not all the results need to be defined, but price of failure should be clear, as well as what will be gained on success.

\subsubsection{Anatomy of challenge}
To collect these things together:
\begin{enumerate}
\item Recognize tense situation where Challenge is required
\item Select ability to use, and how it can overcome the challenge
\item Player and \gM discuss about stakes of the conflict
\item \GM{} can assign circumstansial penalties
\item Player can use existing effects, pool or some other abilities to gain bonus dice
\item Player makes decision, knowing the stakes, whether character will try to perform said action
\item Player rolls dice 3 + bonus/penalty die count
\item Player chooses three best/worst results from dice thrown
\item Result is narrated.
\end{enumerate}

There is also other type of challenge, that is not yet been discussed.
If someone with abilities is resisting the challenge, this challenge resolution is not directly used, instead it is conflict, and will be discussed in next section (Section \ref{ssec:conflict})

\pagebreak

	\subsection{Conflict}\label{ssec:conflict} % myös extended
		\input{system_conflict.tex}
	\subsection{Secrets}\label{ssec:secrets}
		\input{system_secrets.tex}
\section{Character}
	\subsection{Character creation}\label{ssec:creation}
		\input{character_creation.tex}
	\subsection{Mortal}\label{ssec:mortal}
		Mortal characters are good beginning point for any character. 
Mortals in world filled with vampires can be really interesting characters.
These characters can be later updated to vampires, hunters, or thralls as story progresses.

Mortals do not possess unique abilities but can posses unique perspective.
Playing mortal can be great horror experience.
Large amount of vampire fiction is about facing superhuman threat that are out for your blood.

Mortal player characters now typically very little about vampires but more than their peers.
This knowledge puts them them into odds with both vampire and mortal worlds.
Other mortals consider those that know about vampires to be crazy and thus dangerous.
Vampires consider those characters to be nuicanse and possible threat.
However, even though character might know this they cannot return to their previous life, where they were safe as truth is large burden.

Mortal is somehow trying to interact with vampires.
They might have caused death of loved one, or might be mysterious love interest.
Whatever they think about vampires, they are problem.
As they do not know rules that vampires place, they might accidentally release vampire secrets to public.
Worst case might be, that existance of vampires is revealed.
Even asking questions, might be enought to incite violence.

To police, this seems to be stalking and perhaps even attempt on murder.
Even those that mortal convinces, are likely to become ostracised.
What makes things worse, are those vampiric abilities that alter memory, which might start person to question their own sanity.
Also some people these player characters talk to, might be thralls under vampire influence.
Thralls would only try futher isolation of the unwanted individual.


\pagebreak

	\subsection{Vampire}\label{ssec:vampire} % Vampiric keys(mandatory), Humanity and curses(nämä linkataa valittaessa yhteen)
		Vampire player characters are discussed in this chapter.

\subsubsection{Vampiric curses}

As vampire legends go, there are several.
It is up to each group to choose what kind creatures vampires are.
It is commonly known, that vampires have told several legends about themselves.
Some have done this to frighten, hurt other vampires, out of self loathing, or to misdirect.
These are represented with vampiric curses.

There are two ways of acquiring vampiric curses.
All vampires gain some set of curses when they shed mortality and become vampires.
This is not players choice, as it is more of what world does to them.
Other way is to lose \humanity{}, as each \humanity{} is paired with some curse.
That curse becomes part of vampires existance, if they fail to follow that ideal.
This is how vampires become more mosterous as time progresses, and struggle for them to keep their humanity.
List of vampiric curses can be found from \ref{ssec:curses}.

When creating vampire character, in addition to normal process, choose at least one key to represent your vampiric nature.
In addition, you need to choose at least one \humanity{} / vampiric curse pair.
Vampiric curse is not active as long you keept your \humanity{} intact.
If you brake against your \humanity{}, you take Instinct damage depending on bad your violation is.
This represent regret vampire feels for their action.
If vampire does not feel regret (players choice), then that \humanity{} is lost forever.
This can be treated as normal key buyoff.
Vampire should have 1-3 \humanities{}, getting new ones without spending advantages on them.
There is limit to this, you can have total of 7 different \humanities{} during \campaign{}.
Losing last one of these, means that there is nothing human left of the vampire and it has become pure monster.
This can be considered death of the character but player might be allowed to play to end of the current \session{}, showing what kind of monster vampire has become.

\subsubsection{Vampiric secrets}
Vampires posses many mystical abilities.
These are represented with special secrets.
Each vampire secret has tiers.
You need to have lower tier power before you can gain higher tier power.
If you use secret in conflict, you can gain bonus dies.
You can gain benefit of maximum of three die per secret used.
It is also possible that using these abilities evade conflict or give vampire enough leverage challenge otherwise insurmountable threat.
It is very common that mortal might not be able to stand against vampire if they do not posses some special skills to face the vampire.
Activating powers give bonus die equal hightest tier of the power.
You can activate power several time for same conflict, but still the maximum amount can be worth of three dies.

\textit{Example: 
Vampire with Virtus wants to drink from mortal. 
Vampire uses his Virtus and overpowers human who tries to resist.
There is no conflict as mortal has no leverge against Vampire.}

\textit{
Example: 
Vampire with Virtus fights against hunter.
Vampire uses his Virtus and human hunter has prepared for this fight.
There is conflict and both roll. Vampire chooses to use his Virtus(II) and uses two from their Vigor pool in order to get 3 bonus die.}

Some secrets might be inherited through the blood and come easily.
Others should not come easily to vampire.
There should always be reason why vampire is able to learn the secret.
Each secret has some requirement for learning but these should be considered as examples.
This is learning that happens by experiencing something that allows vampire to understand something deep hidden inside them.
In addition to learning by experiencing some drastic moment, vampire might try to find a teacher.
Teach ability might be used to allow vampire to learn secret that teacher possess.
If teacher is not found, vampire might start quest for the secret.
Notice, that this means that even though some secret (e.g. Haruspicy(III)) requirement of some other secret (e.g. Haruspicy(II)) also other methods might be used to acquire it.

\pagebreak
\cols{
\SecretVampiricBase{}
\VampiricAbility{}
\VampiricSecret{}
\VampiricKey{}
}


	\subsection{Elder}\label{ssec:elder} % Vampyyri, joka on niin muinainen, että ei voi esittää samanlaisena kuin tavallisen, esim. 
		Elder vampires, or Elders, are vital part of vampire lore.
As vampires can live indefinitely, some start to become different beings.
These Elders see younger vampires as somewhat lesser beings to guide, patronize, or use.
Elder possesses vast knowledge and skills that only few can equal.
This power can lead isolation for Elder.

% Power difference // that is ok

% What is different to normal vampire

% Inconsistance 

% How to play


\pagebreak

	\subsection{Thrall}\label{ssec:thrall}
		\input{character_thralls.tex}
	\subsection{Hunter}\label{ssec:hunter}
		\input{character_hunters.tex}
\section{World}
	\subsection{Vampiric curses} \label{ssec:curses} % choose mandatory curses
		
Vampiric curses are what define existance of vampire as that of horror.
These both debilitate the vampires and remove them from mortal society.
These curses can effect through supernatural, psychological or sociological.
%% Ei varmaan tartte selittää
%Vampires are entagled with mysticism, so supernatural is easy explanation.
%Sometimes psychosomatic responses can be severe, and so they are if curses exist only in head of vampire.
%Peer pressure and need to belong are basic human needs, and vampire 
Whatever the means are, curses have real tangible effects, even if they come from imagination.

Vampiric curses can be considered to be in layers:
\begin{enumerate}
\item{Universal} 
\item{Setting}
\item{Specific}
\end{enumerate}

Only single curse can be considered universal for vampires in any lore.
This thing defines vampires: lust for blood.
Universal curse is so important, that this book considers it to be always part of vampire.
Other layers are bound to setting.

Setting curse means that these curse comes from setting used.
These come from setting that was chosen when game started.
Typically these are not written in the character sheet, as these should be somewhat common knowledge for players.
If setting is unfamiliar, some printout could be handed to players to help them utilize these. %Todo: hyödyllinen työkalu tätä varten?
Setting curse can be something that applies to every vampire or only some, but it mandated by the setting used.
For example, sunlight might hurt only new vampires or only vampires from certain bloodline.

Specific curses relate to vampires lost humanity.
Each time vampire loses peace of humanity, they also gain curse.
These curses are specific to the vampire, though someone else could have same curses.
Vampires player is one who chooses what curse they gain.
Every player should know these specific curses though as they give interesting opportunities in play.
Player should think what curse is interesting when choosing the curse.
It is better to choose too impactful curse.
If specific curse becomse too debilitating i.e. it removes interesting play instead of adding it, players should discuss about changing it.
In most cases, changing this should not have too big impact, as it can be seen as a phase vampire went through.

Curses presented in this chapter are Setting curses and Specific curses.
Universal curse manifests through certain keys and how pools are refreshed.
Setting curse and Specific curse are not separated.
Any curse presented here can be either.
In some settings no vampire can cross river but in other it affects to just certain vampires.
Curses presented in this book are examples, as there are endless variations on these curses.

These are curses:

\cols{
\CurseSecret{}
}

	\subsection{Vampiric world} \label{ssec:vampire_world}
		%Tässä puhutaan player = ei-pelinjohtaja pelaaja
This chapter is about creating vampire mythos through play.
It might be that you do not have any specific setting in mind and don't want to do research to find correct setting for you.
Perhaps you don't want to force some setting or are interested of creating your own.
This chapter might be just right for you.

Vampires have three components.
They have social component, strengths, and weaknesses.
Social component are shown as natural consequence of the play.
When vampire players interact, that shows how vampires socialize.
Strengths and weaknesses is different matter.
We will delve little deeper into those.

First golden rule is, \textit{if all players agree}, it goes.
When all player agree that something is relevant property of vampire, then it is.
No futher mechanic is needed, though player might want to write it down.
This is very important part, which doe its simplicity does not get much hilighted.
Other ways of inventing new flavor to vampires are little more complex, so they take more space.
But before thinking of applying any other rule, first consider if all players already agree.

Second concept is, that each player is allowed to add as much as they want to the lore.
This books contains examples what to add, but should not be considered as limiting factor.
Only limitation is that these should be choses made these choices as a player, with responisibility related to that.
This means, that this addition is made with intention to make game better.

Lastly, no addition should be made to cancel previously made decisions.
Simply put, when adding new content, it should add, not replace or remove.
This is to prevent nullifying other players intentions.
This rule is difficult to follow, and mistakes can happen.
Simple case for this is if first player adds that vampires start sparkle on moonlight, revealing their presence, 
	second player should not add that vampires can turn invisible in moonlight.
Note that some vampires might have special powers that nullify these curses, 
	but remember that this chapter is about vampires in general.

\subsubsection{Strengths}
This means that learning about uses of this stregth is focal point for the character.
It is represented with related key (drunk of power, OR unable to control).
After this key is bought, the related power is immediatly added part of the vampiric secret which gives benefit for all the characters with that secret.
For player character adding the power, there is related key, that produces some story relating to this power.
If you have characters that are not vampires, they can also evoke this, they just don't get the benefits.

\subsubsection{Weaknesses}
This means that learning about weakness, or having it just start affecting.
This experience is represented with related key (weakness found).
After this key is bought, the related weakness is immediatly added part of the vampiric secret which gives disadvantages for all the characters with that secret.
For player character adding the weakness, there is related key, that produces some story relating to this power.
If you have characters that are not vampires, they can also evoke this, they just don't get the effects,
	Key bought should still be relevant.

\pagebreak

	\subsection{Time passes} \label{ssec:time_passes}
		Common theme that people want to explore with immortality, is passing of time.
This chapter is dedicated to help to explore, when months, years, or centuries pass, while vampire characters remain mostly unchanged. % or stagnant?
This is called time skip, meaning that we move our focus to more interesting time.
Not every game needs to have the passing of time.
During time skip, characters might, for example, learn new skills, consolidate power, heal wounds, build friendships, or make enemies.

First step is to decide that spot is good for doing the time skip.
It might come naturally but normally it requires that everyone on the table knows that there is going to be a time skip.
Good spot is after something major or intense has been just resolved.
The time skip should happen when there is still things ongoing but not immediately at least for every character.
It is also totally acceptable that someone asks that should we start a time skip, and someone respons not yet. % TODO: example?

After deciding that it is good time for a time skip, next it should be decided where and when time skip will take a place.
You can be open discussion but it can also be more interesting to let players make this decision. % erottelu player storyteller?
One inspirational way to do it, is to have one player to make decision on time, 
	and other player makes decision on where characters meet.
When deciding time, player can give more details on time, general stuff that everyone in the world knows.
When deciding location, description of scene and reason why characters meet there should be given.

\textit{First player want that where time skip ends is in 1806 AD. They want to specify that this is date is important as inquisition is abolished.} % TODO: better example?
\textit{Second player want that play will continue in Rome. They tell, that vampires are gathering there, as vampire ruler of Rome is going to make important declaration.}

Now players know how much time is covered with the time skip.
It might be months, or it might be years.
Vampires do not need to just be passive during this time.
Players describe what their characters were doing during this time.
This does not need to be extremely detailed, details can be figured out later.
Only important things should be outlined during this time.
I would recommend just specifying one thing but this is not strict sometimes there are several important things.
These typically cause skill check for each player character. % TODO: skill check?
% TODO: add example

As years pass, characters can also progress.
This is represented with gaining new secrets (Section~\ref{ssec:secrets}).
Normal rules of acquiring secrets still apply.
Secret gaining:
\begin{table}[h]
\begin{tabular}{ll}
Month & 0 secret \\
Year & 2 secret \\
Decade & 4 secret \\
Century & 6 secret \\
Millenia & 8 secret
\end{tabular}
\end{table}

If people are not certain what secrets to buy, these can also be left undecided, and chosen when they would come up in story.
When these are chosen after time skip, it would be good to hear story, where character learned this secret
   \textit{(e.g. When Isabella was in France hiding from wrath of the bishop, she learned to synergize her Vagrancy with other skills) }.
Main purpose of this rule is to not rush people to make bad decisions.
Some secrets might difficult to acquire, and should be done immediatetly, and might require explanation how those are gained.
   \textit{(e.g. Isabella wanted to learn how to read people minds. That is reason why she went to the bishop, only person she knew that would accept money as payment ) }.
Details should are decided on table.



\comment{
%\AllSecrets{}

\begin{adjustwidth}{-1.25cm}{-1.25cm}
\begin{multicols} {3}
%\AllSecret{}
%\AllKey{}
\end{multicols}
\end{adjustwidth}
\section{Termist\"o}
\section{Esittely}
	\subsection{Kirjasta}
		\blindtext
	\subsection{Vampyyreista} % Vampiric curses, secrets
		\blindtext
	\subsection{Roolipelaamisesta}
		\blindtext
	\subsection{Terminologia}
		\blindtext
\section{Maailma}
	\subsection{Vampyyrien merkitys}
	\subsection{Vampyyrien ominaisuudet}
	\subsection{Vampyyri mytologian luonti}
\section{Pelaaminen}
	\subsection{Yleiskuva}
	\subsection{Pelin kulku}
	\subsection{Kyvyt}
	\subsection{Avaimet}
	\subsection{Salaisuudet}
	\subsection{Kykytesti}
	\subsection{Konflikti}
\section{Hahmo}
	\subsubsection{Perus Kuolevainen}
	\subsubsection{Vampyyri}
	\subsubsection{Vanha vampyyri}
	\subsubsection{Muinainen vampyyri}
	\subsubsection{Orja(Thrall) ja Metsästäjä}
	\subsection{Hahmolomake}
	\subsection{Hahmonluonti}
\section{Pelisessio}
	\subsection{Kykytesti}
	\subsection{Konflikti}
	\subsection{Valineet}
	\subsection{Ylosnousemus}
\section{Mekaniikka}
	\subsection{Kyvyt}
		\subsubsection{Innate Abilities}
			\paragraph{Endure}
			\paragraph{React}
			\paragraph{Resist}
		\subsubsection{Vampric Abilities}
			\paragraph{Quell the Beast}
			\paragraph{Strength}
			\paragraph{Mysticism}
			\paragraph{Domination}
		\subsubsection{Combat Abilities (Against)}
			\paragraph{Vampires}
			\paragraph{Hunters}
			\paragraph{War}
			\paragraph{Ranged}
			\paragraph{Brawl}
	\subsection{Avaimet}
		\subsubsection{Ihmisyys}
		\subsubsection{Vampyyrinen}
		\subsubsection{Yleiset}
	\subsection{Derangements}
	\subsection{Aikakausi}
	\subsection{Bloodline}
}
\end{document}

\newcommand{\topic}[1]{}
