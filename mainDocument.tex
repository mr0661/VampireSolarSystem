\documentclass[a4paper, 12pt, finnish]{report}

\usepackage{bookman}
\usepackage{calc}
\usepackage[utf8]{inputenc}
\usepackage[table]{xcolor}
\usepackage{chngpage}
\usepackage{multicol}

\setlength{\parindent}{0em}
\setlength{\parskip}{1em}

\newcommand{\comment}[1]{}

\begin{document}

\newcommand{\nameSpacerG}[0]{\color{black}\Large{g}}
\newcommand{\nameSpacer}[0]{\makebox[0pt][l]{\nameSpacerG}}

\newcommand{\nameSize}[1]{\color{white}\small{ \textbf{#1}}}
\newcommand{\nameCommand}[1]{\nameSpacer\nameSize{#1}}

\newcommand{\emptyTable}[0]{
\begin{minipage}[b]{5cm}
\begin{center}
\begin{tabular}{ p{5cm} } 
 \
\end{tabular}
\end{center}
\end{minipage} \hfill
}

\newcommand{\cols}[1]{
\begin{adjustwidth}{-1.25cm}{-1.25cm}
\begin{multicols} {3}
#1
\end{multicols}
\end{adjustwidth}
}

\newcommand{\abilityMonsterCombat}{
\begin{minipage}[b]{5cm}
\tiny{
\begin{center}
\begin{tabular}{ |p{5cm}| }
\hline
\rowcolor{black}\multicolumn{1}{|c|}{\nameCommand{Monster combat(V)}}
\\
\\[0pt]
This is used when fighting against monsters e.g. vampires. Whenever facing unnatural beast, that uses it supernatural powers those can be turned in to the weakness. It requires that you have prior knowledge of these tricks. It also requires that you have proper tools. These proper tools to counter monsters abilities can be produced with enought time and planning. This is only ability that allows battle against supernatural beings without penalty die. This is also what is used when two monsters fight against each other.
\\[0pt]
\\[0pt]
\hline
\end{tabular}
\end{center}
}
\end{minipage} %%\hspace{-10pt}
}
\newcommand{\abilityMonsterCombat}{
Monster combat (V)
}
\newcommand{\abilityUnarmedCombat}{
\begin{minipage}[b]{5cm}
\tiny{
\begin{center}
\begin{tabular}{ |p{5cm}| }
\hline
\rowcolor{black}\multicolumn{1}{|c|}{\nameCommand{Unarmed combat(V)}}
\\
\\[0pt]
This can only be used when your opponent does not have any weapon. It also requires that opponent fights like normal human being.
\\[0pt]
\\[0pt]
\hline
\end{tabular}
\end{center}
}
\end{minipage} %%\hspace{-10pt}
}
\newcommand{\abilityUnarmedCombat}{
Unarmed combat (V)
}
\newcommand{\abilityAnimalCombat}{
\begin{minipage}[b]{5cm}
\tiny{
\begin{center}
\begin{tabular}{ |p{5cm}| }
\hline
\rowcolor{black}\multicolumn{1}{|c|}{\nameCommand{Animal combat(V)}}
\\
\\[0pt]
This is used when fighting against animal. Animal needs to act like normal animal with normal animal properties. For example, this works against wolfs, horses, dogs, cat, and ravens. It can be used also against something that attacks like an animal. If fighting against something that has unnatural abilities but attacks like animal skill can be used with penalty.
\\[0pt]
\\[0pt]
\hline
\end{tabular}
\end{center}
}
\end{minipage} %%\hspace{-10pt}
}
\newcommand{\abilityAnimalCombat}{
Animal combat (V)
}
\newcommand{\abilityDuel}{
\begin{minipage}[b]{5cm}
\tiny{
\begin{center}
\begin{tabular}{ |p{5cm}| }
\hline
\rowcolor{black}\multicolumn{1}{|c|}{\nameCommand{Duel(V)}}
\\
\\[0pt]
Used when fighting in clear duel situation. This is used mainly to survive in such situations, not necessary to win it. If winning is only way to survive such event, then winning might be what success indicates. It requires that fight is clearly organized and follows some sort of rules. Duels are not typically to the death. This roll is typically contested by your opponent but is not always.
\\[0pt]
\\[0pt]
\hline
\end{tabular}
\end{center}
}
\end{minipage} %%\hspace{-10pt}
}
\newcommand{\abilityDuel}{
Duel (V)
}
\newcommand{\abilityRangedCombat}{
\begin{minipage}[b]{5cm}
\tiny{
\begin{center}
\begin{tabular}{ |p{5cm}| }
\hline
\rowcolor{black}\multicolumn{1}{|c|}{\nameCommand{Ranged combat(V)}}
\\
\\[0pt]
This is used when fighting against opponent that uses ranged weapon. Ability is opposed when fighting with proper weapons, e.g. ranged weapon or with shield. Some other tools or situation might also produce opposed roll. It can also be used while wielding melee weapon but then roll is likely to be parallel and with penalty. It is also possible to use this simply to avoid an attack.
\\[0pt]
\\[0pt]
\hline
\end{tabular}
\end{center}
}
\end{minipage} %%\hspace{-10pt}
}
\newcommand{\abilityRangedCombat}{
Ranged combat (V)
}
\newcommand{\abilityBattle}{
\begin{minipage}[b]{5cm}
\tiny{
\begin{center}
\begin{tabular}{ |p{5cm}| }
\hline
\rowcolor{black}\multicolumn{1}{|c|}{\nameCommand{Battle(V)}}
\\
\\[0pt]
Used when fighting with organized group. This is not for leading army, only participating in combat situation. This is used mainly to survive in such situations, not necessary to win it. In normal situations, this is not contested.
\\[0pt]
\\[0pt]
\hline
\end{tabular}
\end{center}
}
\end{minipage} %%\hspace{-10pt}
}
\newcommand{\abilityBattle}{
Battle (V)
}
\newcommand{\abilityBrawl}{
\begin{minipage}[b]{5cm}
\tiny{
\begin{center}
\begin{tabular}{ |p{5cm}| }
\hline
\rowcolor{black}\multicolumn{1}{|c|}{\nameCommand{Brawl(V)}}
\\
\\[0pt]
Used when fighting in situation that has large group fighting without coorination. This is used mainly to survive in such situations, not necessary to win it. In normal situations, this is not contested.
\\[0pt]
\\[0pt]
\hline
\end{tabular}
\end{center}
}
\end{minipage} %%\hspace{-10pt}
}
\newcommand{\abilityBrawl}{
Brawl (V)
}
\newcommand{\abilityMemory}{
\begin{minipage}[b]{5cm}
\tiny{
\begin{center}
\begin{tabular}{ |p{5cm}| }
\hline
\rowcolor{black}\multicolumn{1}{|c|}{\nameCommand{Memory(R)}}
\\
\\[0pt]
Memory is skill that many value. It is about recalling something. Used when it is not clear that someone could remember such detail. However, everything that is stated to happen is assumed to be in such importance, that characters remember them for rest of their lives. Memory can be used to bring interesting facts into the game. Player could ask that when new NPC is introduced to remember something about them. If they succeed, they remember, and such fact should be generated in a spot. For example, character might remember that same phrase was used by her nemesis and can now suspect that this character knows her nemesis.
\\[0pt]
\\[0pt]
\hline
\end{tabular}
\end{center}
}
\end{minipage} %%\hspace{-10pt}
}
\newcommand{\abilityMemory}{
Memory (R)
}
\newcommand{\abilitySports}{
\begin{minipage}[b]{5cm}
\tiny{
\begin{center}
\begin{tabular}{ |p{5cm}| }
\hline
\rowcolor{black}\multicolumn{1}{|c|}{\nameCommand{Sports(V)}}
\\
\\[0pt]
This used all manner of sport. It is about strenth possessed. Typically used in general athletic performances. This can be when swimming, sprinting, falling etc.
\\[0pt]
\\[0pt]
\hline
\end{tabular}
\end{center}
}
\end{minipage} %%\hspace{-10pt}
}
\newcommand{\abilitySports}{
Sports (V)
}
\newcommand{\abilitySpeak}{
\begin{minipage}[b]{5cm}
\tiny{
\begin{center}
\begin{tabular}{ |p{5cm}| }
\hline
\rowcolor{black}\multicolumn{1}{|c|}{\nameCommand{Speak(R)}}
\\
\\[0pt]
Speak is about explaining yourself. This can be used when you are arguing. Sometimes it takes a form of a inspiring speach. Speak is used when people are reasonable and are willing to listen for it. Witty banter can be in speak but is usually reserced for Charm and Deceit
\\[0pt]
\\[0pt]
\hline
\end{tabular}
\end{center}
}
\end{minipage} %%\hspace{-10pt}
}
\newcommand{\abilitySpeak}{
Speak (R)
}
\newcommand{\abilityBarter}{
\begin{minipage}[b]{5cm}
\tiny{
\begin{center}
\begin{tabular}{ |p{5cm}| }
\hline
\rowcolor{black}\multicolumn{1}{|c|}{\nameCommand{Barter(I)}}
\\
\\[0pt]
Bartering is something that every human has done. This is typically done when trying to buy something. It is both understanding value of something as it is to have someone to miss understand value. Though it is commonly used when purchasing goods, skilled barterer could convince bandits that he is more valuable alive than he is dead. Use this to acquire something.
\\[0pt]
\\[0pt]
\hline
\end{tabular}
\end{center}
}
\end{minipage} %%\hspace{-10pt}
}
\newcommand{\abilityBarter}{
Barter (I)
}
\newcommand{\abilityDeceit}{
\begin{minipage}[b]{5cm}
\tiny{
\begin{center}
\begin{tabular}{ |p{5cm}| }
\hline
\rowcolor{black}\multicolumn{1}{|c|}{\nameCommand{Deceit(I)}}
\\
\\[0pt]
Deceit is art of making people believe you. It is about creating impression that something is the truth. This archieved with many tricks, e.g. giving misleading information, lying, or telling the truth. Deceit does not necessary need to be about giving false information, though it is what most people use it. Sometimes it can be using weakness of mind against itself and help to person understand the reality. As the tricks are same, ability is the same.
\\[0pt]
\\[0pt]
\hline
\end{tabular}
\end{center}
}
\end{minipage} %%\hspace{-10pt}
}
\newcommand{\abilityDeceit}{
Deceit (I)
}
\newcommand{\abilityCharm}{
\begin{minipage}[b]{5cm}
\tiny{
\begin{center}
\begin{tabular}{ |p{5cm}| }
\hline
\rowcolor{black}\multicolumn{1}{|c|}{\nameCommand{Charm(I)}}
\\
\\[0pt]
Charm is about being liked. This means making good first impressions and eventually friends. It is also about manipulation with subtle social means. Romantic conquests are also done with charm.
\\[0pt]
\\[0pt]
\hline
\end{tabular}
\end{center}
}
\end{minipage} %%\hspace{-10pt}
}
\newcommand{\abilityCharm}{
Charm (I)
}
\newcommand{\AllKeyName}{
\KeyNameExampleKeyTwo{}
\KeyNameExampleKey{}
}
\newcommand{\AllKey}{
\KeyExampleKeyTwo{}
\KeyExampleKey{}
}
\newcommand{\KeyExampleKeyTwo}{
\begin{minipage}[b]{5cm}
\tiny{
\begin{center}
\begin{tabular}{ |p{5cm}| }
\hline
\rowcolor{black}\multicolumn{1}{|c|}{\nameCommand{My key 2}}
\\
\\[0pt]
Core text starts here Core text starts here Core text starts here
\vspace{-1mm}
\begin{itemize}
\setlength\itemsep{-2pt}
\item[1xp:] Something minor
%\item[2xp:] :xp2:
\item[3xp:] Something important Something important Something important Something important Something important
%\item[5xp:] :xp5:
\item[buyoff:] Some twist
\end{itemize}
\\
%\multicolumn{1}{|r p{4cm}|}{xp1: & cp1} %%& \multicolumn{1}{p{4cm}|}{Something minor} \\
%%\multicolumn{1}{|r}{xp2:} & \multicolumn{1}{l|}{:xp2:} \\
%\multicolumn{1}{|r}{xp3:} & \multicolumn{1}{l|}{Something important Something important Something important Something important Something important} \\
%%\multicolumn{1}{|r}{xp5:} & \multicolumn{1}{l|}{:xp5:} \\
%\multicolumn{1}{|r}{buyoff:} & \multicolumn{1}{l|}{Some twist} \\
\hline
\end{tabular}
\end{center}
}
\end{minipage}
}
\newcommand{\KeyNameExampleKeyTwo}{
My key 2
}
\newcommand{\KeyExampleKey}{
\begin{minipage}[b]{5cm}
\tiny{
\begin{center}
\begin{tabular}{ |p{5cm}| }
\hline
\rowcolor{black}\multicolumn{1}{|c|}{\nameCommand{My key}}
\\
\\[0pt]
Core text starts here Core text starts here Core text starts here
\vspace{-1mm}
\begin{itemize}
\setlength\itemsep{-2pt}
\item[1xp:] Something minor
\item[2xp:] Something larger
%\item[3xp:] :xp3:
\item[5xp:] Something major
\item[buyoff:] Some twist
\end{itemize}
\\
%\multicolumn{1}{|r p{4cm}|}{xp1: & cp1} %%& \multicolumn{1}{p{4cm}|}{Something minor} \\
%\multicolumn{1}{|r}{xp2:} & \multicolumn{1}{l|}{Something larger} \\
%%\multicolumn{1}{|r}{xp3:} & \multicolumn{1}{l|}{:xp3:} \\
%\multicolumn{1}{|r}{xp5:} & \multicolumn{1}{l|}{Something major} \\
%\multicolumn{1}{|r}{buyoff:} & \multicolumn{1}{l|}{Some twist} \\
\hline
\end{tabular}
\end{center}
}
\end{minipage}
}
\newcommand{\KeyNameExampleKey}{
My key
}
\newcommand{\AllSecrets}{
\secretmyTopic{}
\secretConditioning{}
}
\newcommand{\secretmyTopic}{
\tiny{
\begin{center}
\begin{tabular}{ |p{5cm}| }
\hline
\rowcolor{black}
\multicolumn{1}{|c|}{\nameCommand{something}}
\\
\\[0pt]
Core text starts here Core text starts here Core text starts here
 \textbf{Requirements:} own other secret
% \textbf{Requirements:} :requirements:
 \textbf{Cost:} 1 from any pool
\\[0pt]
\\[0pt]
\hline

\end{tabular}
\end{center}
}
}
\newcommand{\secretConditioning}{
\tiny{
\begin{center}
\begin{tabular}{ |p{5cm}| }
\hline
\rowcolor{black}
\multicolumn{1}{|c|}{\nameCommand{Secret of conditioning}}
\\
\\[0pt]
You are specifically conditioned to use one of your pools when possible. Whenever you spend from specified pool add one additianal bonus die.
% \textbf{Requirements:} :req:
% \textbf{Requirements:} :requirements:
 \textbf{Cost:} 1 extra from chosen pool
\\[0pt]
\\[0pt]
\hline

\end{tabular}
\end{center}
}
}

%% ABILITIES %%

\textbf{Passive abilities:}

\textbf{Passive abilities} are universal to every human and can only be utilized as defensive.
However, almost any time person wants to be defensive, they can choose to use passive ability.

\cols{
\PassiveAbility{}
}

\textbf{Universal abilities:}

\textbf{Universal abilities} are universal to humans in any time or place.

\cols{
\UniversalAbility{}
}

\pagebreak

\textbf{Combat abilities:}

\textbf{Combat abilities} can be are used to defeat opponent with sheer force. In general sense, ability describes how well can you fight against an opponent of specific type or survival in situations. Some combat abilities are general while other are universal. Abilities with combat in name are to defeat opponent of that kind and others are survival in combat situations.


\cols{
\CombatAbility{}
}

\pagebreak

\textbf{General abilities:}

\textbf{General abilities} can be acquire during any time period but might be influenced by culture. These abilties are not possed by every one. General abilities are: \GeneralAbilityName{}

\cols{
\GeneralAbility{}
}

\pagebreak

\textbf{Era abilities:}

\textbf{Era abilities} are useful and acquirable only in certain eras. These are mostly just examples, as there can be quite a lot of these: \EraAbilityName{}

\cols{
\EraAbility{}
}

\pagebreak

\textbf{Vampiric abilities:}

\textbf{Vampiric abilities} are useful and acquirable only in certain eras. These are mostly just examples, as there can be quite a lot of these: \VampiricAbilityName{}

\cols{
\VampiricAbility{}
}

\pagebreak

%% SECRETS %%

\textbf{Universal secrets:}

\textbf{Universal secrets} general secrets that can be considerd to bring some specific aspect of character.

\cols{
\UniversalSecret{}
}

\textbf{Status secrets:}

\textbf{Status secrets} are secrets that nomite some extremely important status. 
Person with those attributes do not need to have mathcing secrets if they do not expect to gain any banafits form them.
Having relevant status secret might give bonus die on some situations or automatic success on others.
Sometimes you might need to have secret to even try.
It is not necessary to have several status secrets if all apply, just choose most describing then (Wealthy king would might be Nobility where rich baron might be Wealthy).
However, if these are unrelated to each other, for example character is vampiric nobility and french political leader, it might make sense to have both.

\cols{
\StatusSecret{}
}


\textbf{Culture secrets:}

\textbf{Culture secrets} relate to specific culture.
To learn these you typically need teacher.
These are needed to not seem foreing in those cultures.
In general sense, languages are acquired through this but some basic knowledge on these can be acquired without need to buy secrets.
For example, it might be typical that people in some culture know several languages.
Then person with that background would also know several languages.
In almost every situation, two person who meet are able to communicate with each other.
In addition, two people from foreign culture are able to communicate with each other without outsider understanding (though there might be exeptions).
With these secret, you do not need to use \AbilityNameDeceit{} on people so that they would believe you to be one of them.

\cols{
\CultureSecret{}
}

\textbf{Ability related secrets:}

\textbf{Ability related secrets} augment uses of some abilities. 
These do not perform miracles instead allow user to utize their abilities.
Many of these secrets will take time and perhaps money to use.

\cols{
\AbilitySecret{}
}



\textbf{Vampiric secrets:}

\textbf{Vampiric secrets} represent powers vampires posses.

\textbf{Vampiric strengths } is vigor based ability. 
Typically secrets are used in combat but might have uses also elsewhere.
If used in combat, vampire is then considered as a Monster.


\cols{
\VampiricStrengthSecret{}
}

\textbf{Vampiric mysticism } is instinct based ability.
These secret provide access to some reality shaping abilities vampires can posses.
These are strong and difficult to use.


\cols{
\VampiricMysticismSecret{}
}

\textbf{Vampiric mind control }

\cols{
\VampiricMindControlSecret{}
}

\pagebreak

%% KEYS %%

\textbf{Key}

Most important and difficult key is key of humanity. It consist of several steps where you can lose humanity slowly. 
Each vampire has set of principles.
More you have, more easily you can function in society.
Maximum number is 7 and minimum is 1.
If all principles are lost, vampire loses its mind.
As more and more principles are lost vampire becomes more obsessive on the remaining ones.
It is possible to develop other principles but whenever principle is lost it cannot be replaced.

\begin{itemize}
\setlength\itemsep{-2pt}
\item[7] normal vampiric curses still apply
\item[6-] it costs single point from any pool to wake up. If you do can't, you slumber
\item[5-] it costs 1 Vigor to appear as a living, to have bodily functions as a living
\item[4-] you acquire some mental illness
\item[3-] you are unable to eat regular food or drink
\item[2-] you cannot consume animal blood
\item[1] you cannot spend Vigor anymore to appear as a human.
\end{itemize}

Example principles are:
\begin{itemize}
\setlength\itemsep{-2pt}
\item Kindness towards others

\item Do onto others as you would have them to do to you
\item Justice and fairness
\item Right of everyone to decide their own fate
\item Right for everyone to their own bodies
\item Respect for human life
\item Integrity of your own body
\end{itemize}

You can write your own. These can also be changed between sessions. However, if one is lost it cannot be replaced. It will be marked as broken principle and it is permanent loss.

Key of Humanity is linked with these. As these describe principles, Key of Humanity is how these can be lost.

\KeyHumanity{}

\KeyNameHumanity{} speaks of remorse. If you break a principle you can:
\begin{itemize}
\item[a.] Buyoff principle. Cross over one principle that you broke with an action, take 10 experience. Vampire will always have \KeyNameHumanity{}
\item[b.] Feel remorse. Take Reason damage for each principle you broke. Each broken principle can contribute between 1 - 3 points of damage. If damage would be larger than largest damage track remorse is impossible.
\end{itemize}

If principles are changed it means that the new thing is actually the real principle and has always been, other has been self deception.
If vampire has broken new principle he has to feel remorse from each of those missteps.
Braking principle should be informed decision from players.
It should not be clear accident.
However, it might be something that player did not want to happen.

\textit{For example, new vampire is unable to control his bloodlust. He therfore tries to drink his beloved forcefully. This is does not respect right that everyone has to their own bodies or respect for life. However as this would be unintentional, it is deemed that first contributes 2 point and second only 1. This means that if this happens, it either cause buyoff or remorse worth of 3 points.
Player has no ability to fight against this kind of bloodlust knowing what the stake (in addition of the beloved).}

\comment{
%\AllSecrets{}

\begin{adjustwidth}{-1.25cm}{-1.25cm}
\begin{multicols} {3}
\AllSecret{}
\AllKey{}
\end{multicols}
\end{adjustwidth}
\section{Termist\"o}
\section{Esittely}
	\subsection{Kirjasta}
		\blindtext
	\subsection{Vampyyreista}
		\blindtext
	\subsection{Roolipelaamisesta}
		\blindtext
	\subsection{Terminologia}
		\blindtext
\section{Maailma}
	\subsection{Vampyyrien merkitys}
	\subsection{Vampyyrien ominaisuudet}
\section{Hahmo}
	\subsection{Hahmotyypit}
		\subsubsection{Vampyyri}
		\subsubsection{Kuolevainen}
		\subsubsection{Orja(Thrall)}
	\subsection{Hahmolomake}
		\subsubsection{kyvyt (abilities)}
		\subsubsection{altaat (Pool \& Pool refreshment)}
		\subsubsection{avaimet (keys)}
		\subsubsection{salaisuudet (secret)}
	\subsection{Hahmonluonti}
\section{Pelisessio}
	\subsection{Kykytesti}
	\subsection{Konflikti}
	\subsection{Valineet}
	\subsection{Ylosnousemus}
\section{Mekaniikka}
	\subsection{Kyvyt}
		\subsubsection{Innate Abilities}
			\paragraph{Endure}
			\paragraph{React}
			\paragraph{Resist}
		\subsubsection{Vampric Abilities}
			\paragraph{Quell the Beast}
			\paragraph{Strength}
			\paragraph{Mysticism}
			\paragraph{Domination}
		\subsubsection{Combat Abilities (Against)}
			\paragraph{Vampires}
			\paragraph{Hunters}
			\paragraph{War}
			\paragraph{Ranged}
			\paragraph{Brawl}
	\subsection{Avaimet}
		\subsubsection{Ihmisyys}
		\subsubsection{Vampyyrinen}
		\subsubsection{Yleiset}
	\subsection{Derangements}
	\subsection{Aikakausi}
	\subsection{Bloodline}
}
\end{document}

