\documentclass[a4paper, 12pt, finnish]{article}

\usepackage{bookman}
\usepackage{calc}
\usepackage[]{hyperref}
\usepackage[utf8]{inputenc}
\usepackage[table]{xcolor}
\usepackage{chngpage}
\usepackage{multicol}
\usepackage{enumitem}

\usepackage[english]{babel}
\usepackage{blindtext}


\setlength{\parindent}{0em}
\setlength{\parskip}{1em}
\setlength{\footskip}{100pt}


\newcommand{\comment}[1]{}

\begin{document}

\newcommand{\nameSpacerG}[0]{\color{black}\Large{g}}
\newcommand{\nameSpacer}[0]{\makebox[0pt][l]{\nameSpacerG}}

\newcommand{\nameSize}[1]{\color{white}\small{ \textbf{#1}}}
\newcommand{\nameCommand}[1]{\nameSpacer\nameSize{#1}}

\newcommand{\emptyTable}[0]{
\begin{minipage}[b]{5cm}
\begin{center}
\begin{tabular}{ p{5cm} } 
 \
\end{tabular}
\end{center}
\end{minipage} \hfill
}

\newcommand{\cols}[1]{
\begin{adjustwidth}{-1.25cm}{-1.25cm}
\begin{multicols} {3}
#1
\end{multicols}
\end{adjustwidth}
}
\newcommand{\humanity}[0]{Moral}
\newcommand{\Humanity}[0]{Moral}
\newcommand{\humanities}[0]{Morals}
\newcommand{\Humanities}[0]{Morals}

\newcommand{\campaign}[0]{campaign}
\newcommand{\Campaign}[0]{Campaign}
\newcommand{\session}[0]{session}
\newcommand{\Session}[0]{Session}

\newcommand{\EraAbilityName}{
\AbilityNameAcademic{}
\AbilityNameCrafting{}
\AbilityNameEngineering{}
\AbilityNameComputer{}
\AbilityNameInvesting{}
\AbilityNameReligion{}
}
\newcommand{\AllAbilityName}{
\AbilityNameHardWork{}
\AbilityNameVehicleMaintenance{}
\AbilityNameFreeload{}
\AbilityNameSkullduggery{}
\AbilityNameMusic{}
\AbilityNameVagrancy{}
\AbilityNameExpression{}
\AbilityNameMedicine{}
\AbilityNameSeduce{}
\AbilityNameOutdoor{}
\AbilityNameStreetwise{}
\AbilityNamePiloting{}
\AbilityNameOccult{}
\AbilityNameSleightOfHand{}
\AbilityNameResearch{}
\AbilityNameTeach{}
\AbilityNameStorytelling{}
\AbilityNameRangedCombat{}
\AbilityNameAnimalCombat{}
\AbilityNameUnarmedCombat{}
\AbilityNameDuel{}
\AbilityNameBattle{}
\AbilityNameBrawl{}
\AbilityNameMonsterCombat{}
\AbilityNameVampiricStrength{}
\AbilityNameVampiricMindControl{}
\AbilityNameVampiricMysticism{}
\AbilityNameMemory{}
\AbilityNameSports{}
\AbilityNameSpeak{}
\AbilityNameBarter{}
\AbilityNameDeceit{}
\AbilityNameCharm{}
\AbilityNameResist{}
\AbilityNameEndure{}
\AbilityNameReact{}
\AbilityNameAcademic{}
\AbilityNameCrafting{}
\AbilityNameEngineering{}
\AbilityNameComputer{}
\AbilityNameInvesting{}
\AbilityNameReligion{}
}
\newcommand{\VampiricAbilityName}{
\AbilityNameVampiricStrength{}
\AbilityNameVampiricMindControl{}
\AbilityNameVampiricMysticism{}
}
\newcommand{\GeneralAbilityName}{
\AbilityNameHardWork{}
\AbilityNameVehicleMaintenance{}
\AbilityNameFreeload{}
\AbilityNameSkullduggery{}
\AbilityNameMusic{}
\AbilityNameVagrancy{}
\AbilityNameExpression{}
\AbilityNameMedicine{}
\AbilityNameSeduce{}
\AbilityNameOutdoor{}
\AbilityNameStreetwise{}
\AbilityNamePiloting{}
\AbilityNameOccult{}
\AbilityNameSleightOfHand{}
\AbilityNameResearch{}
\AbilityNameTeach{}
\AbilityNameStorytelling{}
}
\newcommand{\ModernAbilityName}{
\AbilityNameAcademic{}
\AbilityNameComputer{}
}
\newcommand{\CombatAbilityName}{
\AbilityNameRangedCombat{}
\AbilityNameAnimalCombat{}
\AbilityNameUnarmedCombat{}
\AbilityNameDuel{}
\AbilityNameBattle{}
\AbilityNameBrawl{}
\AbilityNameMonsterCombat{}
}
\newcommand{\UniversalAbilityName}{
\AbilityNameUnarmedCombat{}
\AbilityNameMemory{}
\AbilityNameSports{}
\AbilityNameSpeak{}
\AbilityNameBarter{}
\AbilityNameDeceit{}
\AbilityNameCharm{}
}
\newcommand{\PassiveAbilityName}{
\AbilityNameResist{}
\AbilityNameEndure{}
\AbilityNameReact{}
}
\newcommand{\EraAbility}{
\AbilityAcademic{}
\AbilityCrafting{}
\AbilityEngineering{}
\AbilityComputer{}
\AbilityInvesting{}
\AbilityReligion{}
}
\newcommand{\AllAbility}{
\AbilityHardWork{}
\AbilityVehicleMaintenance{}
\AbilityFreeload{}
\AbilitySkullduggery{}
\AbilityMusic{}
\AbilityVagrancy{}
\AbilityExpression{}
\AbilityMedicine{}
\AbilitySeduce{}
\AbilityOutdoor{}
\AbilityStreetwise{}
\AbilityPiloting{}
\AbilityOccult{}
\AbilitySleightOfHand{}
\AbilityResearch{}
\AbilityTeach{}
\AbilityStorytelling{}
\AbilityRangedCombat{}
\AbilityAnimalCombat{}
\AbilityUnarmedCombat{}
\AbilityDuel{}
\AbilityBattle{}
\AbilityBrawl{}
\AbilityMonsterCombat{}
\AbilityVampiricStrength{}
\AbilityVampiricMindControl{}
\AbilityVampiricMysticism{}
\AbilityMemory{}
\AbilitySports{}
\AbilitySpeak{}
\AbilityBarter{}
\AbilityDeceit{}
\AbilityCharm{}
\AbilityResist{}
\AbilityEndure{}
\AbilityReact{}
\AbilityAcademic{}
\AbilityCrafting{}
\AbilityEngineering{}
\AbilityComputer{}
\AbilityInvesting{}
\AbilityReligion{}
}
\newcommand{\VampiricAbility}{
\AbilityVampiricStrength{}
\AbilityVampiricMindControl{}
\AbilityVampiricMysticism{}
}
\newcommand{\GeneralAbility}{
\AbilityHardWork{}
\AbilityVehicleMaintenance{}
\AbilityFreeload{}
\AbilitySkullduggery{}
\AbilityMusic{}
\AbilityVagrancy{}
\AbilityExpression{}
\AbilityMedicine{}
\AbilitySeduce{}
\AbilityOutdoor{}
\AbilityStreetwise{}
\AbilityPiloting{}
\AbilityOccult{}
\AbilitySleightOfHand{}
\AbilityResearch{}
\AbilityTeach{}
\AbilityStorytelling{}
}
\newcommand{\ModernAbility}{
\AbilityAcademic{}
\AbilityComputer{}
}
\newcommand{\CombatAbility}{
\AbilityRangedCombat{}
\AbilityAnimalCombat{}
\AbilityUnarmedCombat{}
\AbilityDuel{}
\AbilityBattle{}
\AbilityBrawl{}
\AbilityMonsterCombat{}
}
\newcommand{\UniversalAbility}{
\AbilityUnarmedCombat{}
\AbilityMemory{}
\AbilitySports{}
\AbilitySpeak{}
\AbilityBarter{}
\AbilityDeceit{}
\AbilityCharm{}
}
\newcommand{\PassiveAbility}{
\AbilityResist{}
\AbilityEndure{}
\AbilityReact{}
}
\newcommand{\AbilityHardWork}{
\begin{minipage}[b]{5cm}
\scriptsize{
\begin{center}
\begin{tabular}{ |p{5cm}| }
\hline
\rowcolor{black}\multicolumn{1}{|c|}{\nameCommand{Hard work(V)}}
\\
\\[0pt]
This is ability to do the Hard work. Not everyone is is able to do the hard work. Many people in aristrocracy might be totally unable to do manual labor. It is about digging holes, cleaning, or carrying heavy object etc. This is things that do not require any specific skill, only ability to do what is obvious. Used typically when you need to help someone by doing some grunt work.
\\[0pt]
\\[0pt]
\hline
\end{tabular}
\end{center}
}
\end{minipage} %%\hspace{-10pt}
}
\newcommand{\AbilityNameHardWork}{
Hard work (V)
}
\newcommand{\AbilityVehicleMaintenance}{
\begin{minipage}[b]{5cm}
\scriptsize{
\begin{center}
\begin{tabular}{ |p{5cm}| }
\hline
\rowcolor{black}\multicolumn{1}{|c|}{\nameCommand{Vehicle maintenance(R)}}
\\
\\[0pt]
Vehicles and animals need maintaining. This is used to maintain vehicles. User needs to have proper culture or secret to use this to something. Based on cultere one might be able to pilot horse, car, or carriage. Some other might need secret to gain access. These are uncommon, like elephants or airplane. Also if you do not have culture that would allow use of such vehicles you can instead create secret for those.
\\[0pt]
\\[0pt]
\hline
\end{tabular}
\end{center}
}
\end{minipage} %%\hspace{-10pt}
}
\newcommand{\AbilityNameVehicleMaintenance}{
Vehicle maintenance (R)
}
\newcommand{\AbilityFreeload}{
\begin{minipage}[b]{5cm}
\scriptsize{
\begin{center}
\begin{tabular}{ |p{5cm}| }
\hline
\rowcolor{black}\multicolumn{1}{|c|}{\nameCommand{Freeload(I)}}
\\
\\[0pt]
Some people do not contribute. What skilled freeloaders do is that people do not necessary think them as freeloaders. Expecially their victims. This is about being ablo to smooth of from other people. It is traditionally used by poorer aristocrats, artists, incompetents, and alcoholics. Ability is still the same, beeing able to suck up living.
\\[0pt]
\\[0pt]
\hline
\end{tabular}
\end{center}
}
\end{minipage} %%\hspace{-10pt}
}
\newcommand{\AbilityNameFreeload}{
Freeload (I)
}
\newcommand{\AbilitySkullduggery}{
\begin{minipage}[b]{5cm}
\scriptsize{
\begin{center}
\begin{tabular}{ |p{5cm}| }
\hline
\rowcolor{black}\multicolumn{1}{|c|}{\nameCommand{Skullduggery(I)}}
\\
\\[0pt]
Skullduggery is ability to perform criminal activity without getting caught. This is general criminal activity such as mugging, kidnapping, robbery, arson etc.. This is similar to \AbilityNameHardWork{} expect this allows one to do dishonest work. Some secrets might allow one to use \AbilityNameSkullduggery{} to perform more skill based work such as counterfeiting.
\\[0pt]
\\[0pt]
\hline
\end{tabular}
\end{center}
}
\end{minipage} %%\hspace{-10pt}
}
\newcommand{\AbilityNameSkullduggery}{
Skullduggery (I)
}
\newcommand{\AbilityMusic}{
\begin{minipage}[b]{5cm}
\scriptsize{
\begin{center}
\begin{tabular}{ |p{5cm}| }
\hline
\rowcolor{black}\multicolumn{1}{|c|}{\nameCommand{Music(I)}}
\\
\\[0pt]
Music is about ability to play instruments. Instuments vary by culture and this skill can be used to play any that is typical to you culture. It might be piano, or it can be throat singing. It usage is subtle. There is rarely conflicts where music can be used but in general it used to insert emotion to people or to impress.
\\[0pt]
\\[0pt]
\hline
\end{tabular}
\end{center}
}
\end{minipage} %%\hspace{-10pt}
}
\newcommand{\AbilityNameMusic}{
Music (I)
}
\newcommand{\AbilityVagrancy}{
\begin{minipage}[b]{5cm}
\scriptsize{
\begin{center}
\begin{tabular}{ |p{5cm}| }
\hline
\rowcolor{black}\multicolumn{1}{|c|}{\nameCommand{Vagrancy(I)}}
\\
\\[0pt]
This is how beggars and drifters survive. It is about gathering what you can to survive. When everything is in dumps, this can be used to keep on living without anything else than clothes on your back. Used when living within civilized people. You for example, you trick, beg, loot, or dumpster dive. You do whatever it takes. Some of these are illigal but all are benign. If you try to maintain dignity by not doing something, take penalty die. If want to do this without braking any laws, take penalty die.
\\[0pt]
\\[0pt]
\hline
\end{tabular}
\end{center}
}
\end{minipage} %%\hspace{-10pt}
}
\newcommand{\AbilityNameVagrancy}{
Vagrancy (I)
}
\newcommand{\AbilityExpression}{
\begin{minipage}[b]{5cm}
\scriptsize{
\begin{center}
\begin{tabular}{ |p{5cm}| }
\hline
\rowcolor{black}\multicolumn{1}{|c|}{\nameCommand{Expression(I)}}
\\
\\[0pt]
Expression is about artistic expression. Used when trying to produce some new art. It can be painting, novel, or sinphony. Creator needs to posses also knowledge how use the medium.
\\[0pt]
\\[0pt]
\hline
\end{tabular}
\end{center}
}
\end{minipage} %%\hspace{-10pt}
}
\newcommand{\AbilityNameExpression}{
Expression (I)
}
\newcommand{\AbilityMedicine}{
\begin{minipage}[b]{5cm}
\scriptsize{
\begin{center}
\begin{tabular}{ |p{5cm}| }
\hline
\rowcolor{black}\multicolumn{1}{|c|}{\nameCommand{Medicine(R)}}
\\
\\[0pt]
Medicine is about healing. Used when taking care of wounded or sick. This is basic care but with proper secrets it can be used to achieve more. On its own, it could be thought as something nurse might do.
\\[0pt]
\\[0pt]
\hline
\end{tabular}
\end{center}
}
\end{minipage} %%\hspace{-10pt}
}
\newcommand{\AbilityNameMedicine}{
Medicine (R)
}
\newcommand{\AbilitySeduce}{
\begin{minipage}[b]{5cm}
\scriptsize{
\begin{center}
\begin{tabular}{ |p{5cm}| }
\hline
\rowcolor{black}\multicolumn{1}{|c|}{\nameCommand{Seduce(I)}}
\\
\\[0pt]
Seduce is not about romance, it is promises of pleasure. It can also be about delivering on those promises. Some skilled people make this look like form of art. \AbilityNameCharm{} is similar but wider. This is more narrow but result is more direct.
\\[0pt]
\\[0pt]
\hline
\end{tabular}
\end{center}
}
\end{minipage} %%\hspace{-10pt}
}
\newcommand{\AbilityNameSeduce}{
Seduce (I)
}
\newcommand{\AbilityOutdoor}{
\begin{minipage}[b]{5cm}
\scriptsize{
\begin{center}
\begin{tabular}{ |p{5cm}| }
\hline
\rowcolor{black}\multicolumn{1}{|c|}{\nameCommand{Outdoor(R)}}
\\
\\[0pt]
Outdoor is about surving in the wild. It is about finding sustenance, place to sleep, and where you want to go. Doing these things safely is also key part. Someone with \AbilityNameStorytelling{} might know where the north is but that would not make journey safe.
\\[0pt]
\\[0pt]
\hline
\end{tabular}
\end{center}
}
\end{minipage} %%\hspace{-10pt}
}
\newcommand{\AbilityNameOutdoor}{
Outdoor (R)
}
\newcommand{\AbilityStreetwise}{
\begin{minipage}[b]{5cm}
\scriptsize{
\begin{center}
\begin{tabular}{ |p{5cm}| }
\hline
\rowcolor{black}\multicolumn{1}{|c|}{\nameCommand{Streetwise(I)}}
\\
\\[0pt]
Some people know how to workout odd place that is the city. This is about blending in, finding right people, or navigating through in cities. Not necessary for everyone living on city or doing for obvious things. It is about new places or difficult situations.
\\[0pt]
\\[0pt]
\hline
\end{tabular}
\end{center}
}
\end{minipage} %%\hspace{-10pt}
}
\newcommand{\AbilityNameStreetwise}{
Streetwise (I)
}
\newcommand{\AbilityPiloting}{
\begin{minipage}[b]{5cm}
\scriptsize{
\begin{center}
\begin{tabular}{ |p{5cm}| }
\hline
\rowcolor{black}\multicolumn{1}{|c|}{\nameCommand{Piloting(V)}}
\\
\\[0pt]
Piloting is extremely general term. It is about using some vehicle of movement. It can be live or inorganic or combination. User needs to have proper culture or secret to use this to something. Based on cultere one might be able to pilot horse, car, or carriage. Some other might need secret to gain access. These are uncommon, like elephants or airplane. Also if you do not have culture that would allow use of such vehicles you can instead create secret for those.
\\[0pt]
\\[0pt]
\hline
\end{tabular}
\end{center}
}
\end{minipage} %%\hspace{-10pt}
}
\newcommand{\AbilityNamePiloting}{
Piloting (V)
}
\newcommand{\AbilityOccult}{
\begin{minipage}[b]{5cm}
\scriptsize{
\begin{center}
\begin{tabular}{ |p{5cm}| }
\hline
\rowcolor{black}\multicolumn{1}{|c|}{\nameCommand{Occult(R)}}
\\
\\[0pt]
There exist odd things in the world. Occult is about knowing about those. It is used to know about mystical things or when researching those. Usually obtained by witches, inquisitors, and conspiricy theorists. Skill determines how well you have found out what is actually true and what is not.
\\[0pt]
\\[0pt]
\hline
\end{tabular}
\end{center}
}
\end{minipage} %%\hspace{-10pt}
}
\newcommand{\AbilityNameOccult}{
Occult (R)
}
\newcommand{\AbilitySleightOfHand}{
\begin{minipage}[b]{5cm}
\scriptsize{
\begin{center}
\begin{tabular}{ |p{5cm}| }
\hline
\rowcolor{black}\multicolumn{1}{|c|}{\nameCommand{Sleight of hand(V)}}
\\
\\[0pt]
Characters with hand dexterity have this. Uses are many but usually needs to be accompanied with related secret. Alone it is used for different kinds juggling.
\\[0pt]
\\[0pt]
\hline
\end{tabular}
\end{center}
}
\end{minipage} %%\hspace{-10pt}
}
\newcommand{\AbilityNameSleightOfHand}{
Sleight of hand (V)
}
\newcommand{\AbilityResearch}{
\begin{minipage}[b]{5cm}
\scriptsize{
\begin{center}
\begin{tabular}{ |p{5cm}| }
\hline
\rowcolor{black}\multicolumn{1}{|c|}{\nameCommand{Research(R)}}
\\
\\[0pt]
When one needs to understand world better they can use \AbilityNameResearch{}. It is used by investigators, natural philosophers, or scientists. It might require time, money, access to libraries, and possible test subjects. Keeping things ethical might cause penalty die.
\\[0pt]
\\[0pt]
\hline
\end{tabular}
\end{center}
}
\end{minipage} %%\hspace{-10pt}
}
\newcommand{\AbilityNameResearch}{
Research (R)
}
\newcommand{\AbilityTeach}{
\begin{minipage}[b]{5cm}
\scriptsize{
\begin{center}
\begin{tabular}{ |p{5cm}| }
\hline
\rowcolor{black}\multicolumn{1}{|c|}{\nameCommand{Teach(R)}}
\\
\\[0pt]
For most of human history, teaching has been extremely valued skill. Can be used to allow someone else to acquire new ability or to buy secret owned by teacher. This takes time failure might be result of teachers skill or that of the student. Nobody can be taught without their consent and some student are easier to teach than others.
\\[0pt]
\\[0pt]
\hline
\end{tabular}
\end{center}
}
\end{minipage} %%\hspace{-10pt}
}
\newcommand{\AbilityNameTeach}{
Teach (R)
}
\newcommand{\AbilityStorytelling}{
\begin{minipage}[b]{5cm}
\scriptsize{
\begin{center}
\begin{tabular}{ |p{5cm}| }
\hline
\rowcolor{black}\multicolumn{1}{|c|}{\nameCommand{Storytelling(R)}}
\\
\\[0pt]
Stories contain small pieces of truth. Storytelling is about knowing stories and being able to recite those. Skilled storytellers know what things stories try to teach and are able to use that specific knowledge. Storytelling is when trying to find what is true about something that could be found in stories.
\\[0pt]
\\[0pt]
\hline
\end{tabular}
\end{center}
}
\end{minipage} %%\hspace{-10pt}
}
\newcommand{\AbilityNameStorytelling}{
Storytelling (R)
}
\newcommand{\AbilityRangedCombat}{
\begin{minipage}[b]{5cm}
\scriptsize{
\begin{center}
\begin{tabular}{ |p{5cm}| }
\hline
\rowcolor{black}\multicolumn{1}{|c|}{\nameCommand{Ranged combat(V)}}
\\
\\[0pt]
This is used when fighting against opponent that uses ranged weapon. Ability is opposed when fighting with proper weapons, e.g. ranged weapon or with shield. Some other tools or situation might also produce opposed roll. It can also be used while wielding melee weapon but then roll is likely to be parallel and with penalty. It is also possible to use this simply to avoid an attack.
\\[0pt]
\\[0pt]
\hline
\end{tabular}
\end{center}
}
\end{minipage} %%\hspace{-10pt}
}
\newcommand{\AbilityNameRangedCombat}{
Ranged combat (V)
}
\newcommand{\AbilityAnimalCombat}{
\begin{minipage}[b]{5cm}
\scriptsize{
\begin{center}
\begin{tabular}{ |p{5cm}| }
\hline
\rowcolor{black}\multicolumn{1}{|c|}{\nameCommand{Animal combat(V)}}
\\
\\[0pt]
This is used when fighting against animal. Animal needs to act like normal animal with normal animal properties. For example, this works against wolfs, horses, dogs, cat, and ravens. It can be used also against something that attacks like an animal. If fighting against something that has unnatural abilities but attacks like animal skill can be used with penalty.
\\[0pt]
\\[0pt]
\hline
\end{tabular}
\end{center}
}
\end{minipage} %%\hspace{-10pt}
}
\newcommand{\AbilityNameAnimalCombat}{
Animal combat (V)
}
\newcommand{\AbilityUnarmedCombat}{
\begin{minipage}[b]{5cm}
\scriptsize{
\begin{center}
\begin{tabular}{ |p{5cm}| }
\hline
\rowcolor{black}\multicolumn{1}{|c|}{\nameCommand{Unarmed combat(V)}}
\\
\\[0pt]
This can only be used when your opponent does not have any weapon. It also requires that opponent fights like normal human being.
\\[0pt]
\\[0pt]
\hline
\end{tabular}
\end{center}
}
\end{minipage} %%\hspace{-10pt}
}
\newcommand{\AbilityNameUnarmedCombat}{
Unarmed combat (V)
}
\newcommand{\AbilityDuel}{
\begin{minipage}[b]{5cm}
\scriptsize{
\begin{center}
\begin{tabular}{ |p{5cm}| }
\hline
\rowcolor{black}\multicolumn{1}{|c|}{\nameCommand{Duel(V)}}
\\
\\[0pt]
Used when fighting in clear duel situation. This is used mainly to survive in such situations, not necessary to win it. If winning is only way to survive such event, then winning might be what success indicates. It requires that fight is clearly organized and follows some sort of rules. Duels are not typically to the death. This roll is typically contested by your opponent but is not always.
\\[0pt]
\\[0pt]
\hline
\end{tabular}
\end{center}
}
\end{minipage} %%\hspace{-10pt}
}
\newcommand{\AbilityNameDuel}{
Duel (V)
}
\newcommand{\AbilityBattle}{
\begin{minipage}[b]{5cm}
\scriptsize{
\begin{center}
\begin{tabular}{ |p{5cm}| }
\hline
\rowcolor{black}\multicolumn{1}{|c|}{\nameCommand{Battle(V)}}
\\
\\[0pt]
Used when fighting with organized group. This is not for leading army, only participating in combat situation. This is used mainly to survive in such situations, not necessary to win it. In normal situations, this is not contested.
\\[0pt]
\\[0pt]
\hline
\end{tabular}
\end{center}
}
\end{minipage} %%\hspace{-10pt}
}
\newcommand{\AbilityNameBattle}{
Battle (V)
}
\newcommand{\AbilityBrawl}{
\begin{minipage}[b]{5cm}
\scriptsize{
\begin{center}
\begin{tabular}{ |p{5cm}| }
\hline
\rowcolor{black}\multicolumn{1}{|c|}{\nameCommand{Brawl(V)}}
\\
\\[0pt]
Used when fighting in situation that has large group fighting without coorination. This is used mainly to survive in such situations, not necessary to win it. In normal situations, this is not contested.
\\[0pt]
\\[0pt]
\hline
\end{tabular}
\end{center}
}
\end{minipage} %%\hspace{-10pt}
}
\newcommand{\AbilityNameBrawl}{
Brawl (V)
}
\newcommand{\AbilityMonsterCombat}{
\begin{minipage}[b]{5cm}
\scriptsize{
\begin{center}
\begin{tabular}{ |p{5cm}| }
\hline
\rowcolor{black}\multicolumn{1}{|c|}{\nameCommand{Monster combat(V)}}
\\
\\[0pt]
This is used when fighting against monsters e.g. vampires. Whenever facing unnatural beast, that uses it supernatural powers those can be turned in to the weakness. It requires that you have prior knowledge of these tricks. It also requires that you have proper tools. These proper tools to counter monsters abilities can be produced with enought time and planning. This is only ability that allows battle against supernatural beings without penalty die. This is also what is used when two monsters fight against each other.
\\[0pt]
\\[0pt]
\hline
\end{tabular}
\end{center}
}
\end{minipage} %%\hspace{-10pt}
}
\newcommand{\AbilityNameMonsterCombat}{
Monster combat (V)
}
\newcommand{\AbilityVampiricStrength}{
\begin{minipage}[b]{5cm}
\scriptsize{
\begin{center}
\begin{tabular}{ |p{5cm}| }
\hline
\rowcolor{black}\multicolumn{1}{|c|}{\nameCommand{Strength(V)}}
\\
\\[0pt]
Vampires are known of their strenght. This ability does not alone do anything but describes how well vampire is able to tap into mystical strentgh they posses.
\\[0pt]
\\[0pt]
\hline
\end{tabular}
\end{center}
}
\end{minipage} %%\hspace{-10pt}
}
\newcommand{\AbilityNameVampiricStrength}{
Strength (V)
}
\newcommand{\AbilityVampiricMindControl}{
\begin{minipage}[b]{5cm}
\scriptsize{
\begin{center}
\begin{tabular}{ |p{5cm}| }
\hline
\rowcolor{black}\multicolumn{1}{|c|}{\nameCommand{Mind control(R)}}
\\
\\[0pt]
Vampires are know to posses ability to insert their will onto others. This ability does not alone do anything but describes how well vampire is able to tap into their mystical ability to affect others.
\\[0pt]
\\[0pt]
\hline
\end{tabular}
\end{center}
}
\end{minipage} %%\hspace{-10pt}
}
\newcommand{\AbilityNameVampiricMindControl}{
Mind control (R)
}
\newcommand{\AbilityVampiricMysticism}{
\begin{minipage}[b]{5cm}
\scriptsize{
\begin{center}
\begin{tabular}{ |p{5cm}| }
\hline
\rowcolor{black}\multicolumn{1}{|c|}{\nameCommand{Mysticism(I)}}
\\
\\[0pt]
Vampires are know to posses supernatural ways to alter reality. This ability does not alone do anything but describes how well vampire is able to tap into their mystical abilities to alter world around them.
\\[0pt]
\\[0pt]
\hline
\end{tabular}
\end{center}
}
\end{minipage} %%\hspace{-10pt}
}
\newcommand{\AbilityNameVampiricMysticism}{
Mysticism (I)
}
\newcommand{\AbilityMemory}{
\begin{minipage}[b]{5cm}
\scriptsize{
\begin{center}
\begin{tabular}{ |p{5cm}| }
\hline
\rowcolor{black}\multicolumn{1}{|c|}{\nameCommand{Memory(R)}}
\\
\\[0pt]
Memory is skill that many value. It is about recalling something. Used when it is not clear that someone could remember such detail. However, everything that is stated to happen is assumed to be in such importance, that characters remember them for rest of their lives. Memory can be used to bring interesting facts into the game. Player could ask that when new NPC is introduced to remember something about them. If they succeed, they remember, and such fact should be generated in a spot. For example, character might remember that same phrase was used by her nemesis and can now suspect that this character knows her nemesis.
\\[0pt]
\\[0pt]
\hline
\end{tabular}
\end{center}
}
\end{minipage} %%\hspace{-10pt}
}
\newcommand{\AbilityNameMemory}{
Memory (R)
}
\newcommand{\AbilitySports}{
\begin{minipage}[b]{5cm}
\scriptsize{
\begin{center}
\begin{tabular}{ |p{5cm}| }
\hline
\rowcolor{black}\multicolumn{1}{|c|}{\nameCommand{Sports(V)}}
\\
\\[0pt]
This used all manner of sport. It is about strenth possessed. Typically used in general athletic performances. This can be when swimming, sprinting, falling etc.
\\[0pt]
\\[0pt]
\hline
\end{tabular}
\end{center}
}
\end{minipage} %%\hspace{-10pt}
}
\newcommand{\AbilityNameSports}{
Sports (V)
}
\newcommand{\AbilitySpeak}{
\begin{minipage}[b]{5cm}
\scriptsize{
\begin{center}
\begin{tabular}{ |p{5cm}| }
\hline
\rowcolor{black}\multicolumn{1}{|c|}{\nameCommand{Speak(R)}}
\\
\\[0pt]
Speak is about explaining yourself. This can be used when you are arguing. Sometimes it takes a form of a inspiring speach. Speak is used when people are reasonable and are willing to listen for it. Witty banter can be in speak but is usually reserced for Charm and Deceit
\\[0pt]
\\[0pt]
\hline
\end{tabular}
\end{center}
}
\end{minipage} %%\hspace{-10pt}
}
\newcommand{\AbilityNameSpeak}{
Speak (R)
}
\newcommand{\AbilityBarter}{
\begin{minipage}[b]{5cm}
\scriptsize{
\begin{center}
\begin{tabular}{ |p{5cm}| }
\hline
\rowcolor{black}\multicolumn{1}{|c|}{\nameCommand{Barter(I)}}
\\
\\[0pt]
Bartering is something that every human has done. This is typically done when trying to buy something. It is both understanding value of something as it is to have someone to miss understand value. Though it is commonly used when purchasing goods, skilled barterer could convince bandits that he is more valuable alive than he is dead. Use this to acquire something.
\\[0pt]
\\[0pt]
\hline
\end{tabular}
\end{center}
}
\end{minipage} %%\hspace{-10pt}
}
\newcommand{\AbilityNameBarter}{
Barter (I)
}
\newcommand{\AbilityDeceit}{
\begin{minipage}[b]{5cm}
\scriptsize{
\begin{center}
\begin{tabular}{ |p{5cm}| }
\hline
\rowcolor{black}\multicolumn{1}{|c|}{\nameCommand{Deceit(I)}}
\\
\\[0pt]
Deceit is art of making people believe you. It is about creating impression that something is the truth. This achieved with many tricks, e.g. giving misleading information, lying, or telling the truth. Deceit does not necessary need to be about giving false information, though it is what most people use it. Sometimes it can be using weakness of mind against itself and help to person understand the reality. As the tricks are same, ability is the same.
\\[0pt]
\\[0pt]
\hline
\end{tabular}
\end{center}
}
\end{minipage} %%\hspace{-10pt}
}
\newcommand{\AbilityNameDeceit}{
Deceit (I)
}
\newcommand{\AbilityCharm}{
\begin{minipage}[b]{5cm}
\scriptsize{
\begin{center}
\begin{tabular}{ |p{5cm}| }
\hline
\rowcolor{black}\multicolumn{1}{|c|}{\nameCommand{Charm(I)}}
\\
\\[0pt]
Charm is about being liked. This means making good first impressions and eventually friends. It is also about manipulation with subtle social means. Romantic conquests are also done with charm.
\\[0pt]
\\[0pt]
\hline
\end{tabular}
\end{center}
}
\end{minipage} %%\hspace{-10pt}
}
\newcommand{\AbilityNameCharm}{
Charm (I)
}
\newcommand{\AbilityResist}{
\begin{minipage}[b]{5cm}
\scriptsize{
\begin{center}
\begin{tabular}{ |p{5cm}| }
\hline
\rowcolor{black}\multicolumn{1}{|c|}{\nameCommand{Resist(R)}}
\\
\\[0pt]
Measures characters strength of will. Is challenged in social settings and during great stress. Useful when one wants to continue on his chosen path.
\\[0pt]
\\[0pt]
\hline
\end{tabular}
\end{center}
}
\end{minipage} %%\hspace{-10pt}
}
\newcommand{\AbilityNameResist}{
Resist (R)
}
\newcommand{\AbilityEndure}{
\begin{minipage}[b]{5cm}
\scriptsize{
\begin{center}
\begin{tabular}{ |p{5cm}| }
\hline
\rowcolor{black}\multicolumn{1}{|c|}{\nameCommand{Endure(V)}}
\\
\\[0pt]
Measures characters endurance and pain tolerance.
\\[0pt]
\\[0pt]
\hline
\end{tabular}
\end{center}
}
\end{minipage} %%\hspace{-10pt}
}
\newcommand{\AbilityNameEndure}{
Endure (V)
}
\newcommand{\AbilityReact}{
\begin{minipage}[b]{5cm}
\scriptsize{
\begin{center}
\begin{tabular}{ |p{5cm}| }
\hline
\rowcolor{black}\multicolumn{1}{|c|}{\nameCommand{React(I)}}
\\
\\[0pt]
Measures characters ability to think and act fast and clearly. It is about understanding situation and acting accordingly.
\\[0pt]
\\[0pt]
\hline
\end{tabular}
\end{center}
}
\end{minipage} %%\hspace{-10pt}
}
\newcommand{\AbilityNameReact}{
React (I)
}
\newcommand{\AbilityAcademic}{
\begin{minipage}[b]{5cm}
\scriptsize{
\begin{center}
\begin{tabular}{ |p{5cm}| }
\hline
\rowcolor{black}\multicolumn{1}{|c|}{\nameCommand{Academics(R)}}
\\
\\[0pt]
This is about being able to work as academic. It contains several different skills but each relate closesly to each other. It is about blending in, getting tenure or grants, teaching monotonous courses, being respected etc..
\\[0pt]
\\[0pt]
\hline
\end{tabular}
\end{center}
}
\end{minipage} %%\hspace{-10pt}
}
\newcommand{\AbilityNameAcademic}{
Academics (R)
}
\newcommand{\AbilityCrafting}{
\begin{minipage}[b]{5cm}
\scriptsize{
\begin{center}
\begin{tabular}{ |p{5cm}| }
\hline
\rowcolor{black}\multicolumn{1}{|c|}{\nameCommand{Crafting (Specific)(I)}}
\\
\\[0pt]
There are many professional crafting skill through history. It should be specified what profession it is as these are not interchangable. This can be used when one needs to act as professional crafter. Specific crafters are, for exampe, pioneer, smith, carpenter, and cable guy.
\\[0pt]
\\[0pt]
\hline
\end{tabular}
\end{center}
}
\end{minipage} %%\hspace{-10pt}
}
\newcommand{\AbilityNameCrafting}{
Crafting (Specific) (I)
}
\newcommand{\AbilityEngineering}{
\begin{minipage}[b]{5cm}
\scriptsize{
\begin{center}
\begin{tabular}{ |p{5cm}| }
\hline
\rowcolor{black}\multicolumn{1}{|c|}{\nameCommand{Engineering (Specific)(R)}}
\\
\\[0pt]
There are many different professional engineers through history. It should be specified what profession it is as these are not interchangable. This can be used when one needs to act as professional engineer. Specific crafters are, for example, different architecths, chemists, and siege engineers.
\\[0pt]
\\[0pt]
\hline
\end{tabular}
\end{center}
}
\end{minipage} %%\hspace{-10pt}
}
\newcommand{\AbilityNameEngineering}{
Engineering (Specific) (R)
}
\newcommand{\AbilityComputer}{
\begin{minipage}[b]{5cm}
\scriptsize{
\begin{center}
\begin{tabular}{ |p{5cm}| }
\hline
\rowcolor{black}\multicolumn{1}{|c|}{\nameCommand{Computer(R)}}
\\
\\[0pt]
This is about using computers. It is surfing in internet and programming. Some secrets might be required to use more obscure computer skills e.g. hacking.
\\[0pt]
\\[0pt]
\hline
\end{tabular}
\end{center}
}
\end{minipage} %%\hspace{-10pt}
}
\newcommand{\AbilityNameComputer}{
Computer (R)
}
\newcommand{\AbilityInvesting}{
\begin{minipage}[b]{5cm}
\scriptsize{
\begin{center}
\begin{tabular}{ |p{5cm}| }
\hline
\rowcolor{black}\multicolumn{1}{|c|}{\nameCommand{Investing(I)}}
\\
\\[0pt]
This is about making correct decision while trying to accumulate wealth. It is used by long time investers and wall street gamblers alike. Can be used to asses chance of success of enterprice.
\\[0pt]
\\[0pt]
\hline
\end{tabular}
\end{center}
}
\end{minipage} %%\hspace{-10pt}
}
\newcommand{\AbilityNameInvesting}{
Investing (I)
}
\newcommand{\AbilityReligion}{
\begin{minipage}[b]{5cm}
\scriptsize{
\begin{center}
\begin{tabular}{ |p{5cm}| }
\hline
\rowcolor{black}\multicolumn{1}{|c|}{\nameCommand{Religion (Specific)(R)}}
\\
\\[0pt]
This is about being able to perform religious rituals and practices. Possessed by most with proper cultural background. Clergy are usually more skilled with it but is possesed by everyone in that believes in that religion. This is many times matter of life and death as inability to perform, understand, or act upon these wants This is not about belief it is about ability to recall texts, argue for inteprations, and recognize those who lack proper understanding. Can be combined with secret to make more potent, for example to preach and convert.
\\[0pt]
\\[0pt]
\hline
\end{tabular}
\end{center}
}
\end{minipage} %%\hspace{-10pt}
}
\newcommand{\AbilityNameReligion}{
Religion (Specific) (R)
}
\newcommand{\AllKeys}{
\keymyTopictt{}
}
\newcommand{\keymyTopictt}{
\tiny{
\begin{center}
\begin{tabular}{ |p{5cm}| }
\hline
\rowcolor{black}\multicolumn{1}{|c|}{\nameCommand{My key}}
\\
\\[0pt]
Core text starts here Core text starts here Core text starts here
\vspace{-3mm}
\begin{itemize}
\setlength\itemsep{-4pt}
\item[1xp:] Something minor
\item[2xp:] Something larger
%\item[3xp:] :xp3:
\item[5xp:] Something major
\item[buyoff:] Some twist
\end{itemize}
\\
%\multicolumn{1}{|r p{4cm}|}{xp1: & cp1} %%& \multicolumn{1}{p{4cm}|}{Something minor} \\
%\multicolumn{1}{|r}{xp2:} & \multicolumn{1}{l|}{Something larger} \\
%%\multicolumn{1}{|r}{xp3:} & \multicolumn{1}{l|}{:xp3:} \\
%\multicolumn{1}{|r}{xp5:} & \multicolumn{1}{l|}{Something major} \\
%\multicolumn{1}{|r}{buyoff:} & \multicolumn{1}{l|}{Some twist} \\
\hline
\end{tabular}
\end{center}
}
}
\newcommand{\AllSecret}{
\SecretExampleSecret{}
\SecretConditioning{}
}
\newcommand{\AllSecretName}{
\SecretNameExampleSecret{}
\SecretNameConditioning{}
}
\newcommand{\SecretExampleSecret}{
\begin{minipage}[b]{5cm}
\tiny{
\begin{center}
\begin{tabular}{ |p{5cm}| }
\hline
\rowcolor{black}
\multicolumn{1}{|c|}{\nameCommand{something}}
\\
\\[0pt]
Core text starts here Core text starts here Core text starts here
 \textbf{Requirements:} own other secret
% \textbf{Requirements:} :requirements:
 \textbf{Cost:} 1 from any pool
\\[0pt]
\\[0pt]
\hline

\end{tabular}
\end{center}
}
\end{minipage}
}
\newcommand{\SecretNameExampleSecret}{
something
}
\newcommand{\SecretConditioning}{
\begin{minipage}[b]{5cm}
\tiny{
\begin{center}
\begin{tabular}{ |p{5cm}| }
\hline
\rowcolor{black}
\multicolumn{1}{|c|}{\nameCommand{Secret of conditioning}}
\\
\\[0pt]
You are specifically conditioned to use one of your pools when possible. Whenever you spend from specified pool add one additianal bonus die.
% \textbf{Requirements:} :req:
% \textbf{Requirements:} :requirements:
 \textbf{Cost:} 1 extra from chosen pool
\\[0pt]
\\[0pt]
\hline

\end{tabular}
\end{center}
}
\end{minipage}
}
\newcommand{\SecretNameConditioning}{
Secret of conditioning
}
\comment{
%% ABILITIES %%

\textbf{Passive abilities:}

\textbf{Passive abilities} are universal to every human and can only be utilized as defensive.
However, almost any time person wants to be defensive, they can choose to use passive ability.

\cols{
\PassiveAbility{}
}

\textbf{Universal abilities:}

\textbf{Universal abilities} are universal to humans in any time or place.

\cols{
\UniversalAbility{}
}

\pagebreak

\textbf{Combat abilities:}

\textbf{Combat abilities} can be are used to defeat opponent with sheer force. In general sense, ability describes how well can you fight against an opponent of specific type or survival in situations. Some combat abilities are general while other are universal. Abilities with combat in name are to defeat opponent of that kind and others are survival in combat situations.


\cols{
\CombatAbility{}
}

\pagebreak

\textbf{General abilities:}

\textbf{General abilities} can be acquire during any time period but might be influenced by culture. These abilties are not possed by every one. General abilities are: \GeneralAbilityName{}

\cols{
\GeneralAbility{}
}

\pagebreak

\textbf{Era abilities:}

\textbf{Era abilities} are useful and acquirable only in certain eras. These are mostly just examples, as there can be quite a lot of these: \EraAbilityName{}

\cols{
\EraAbility{}
}

\pagebreak

\textbf{Vampiric abilities:}

\textbf{Vampiric abilities} are useful and acquirable only in certain eras. These are mostly just examples, as there can be quite a lot of these: \VampiricAbilityName{}

\cols{
\VampiricAbility{}
}

\pagebreak

%% SECRETS %%

\textbf{Universal secrets:}

\textbf{Universal secrets} general secrets that can be considerd to bring some specific aspect of character.

\cols{
\UniversalSecret{}
}

\textbf{Status secrets:}

\textbf{Status secrets} are secrets that nomite some extremely important status. 
Person with those attributes do not need to have mathcing secrets if they do not expect to gain any banafits form them.
Having relevant status secret might give bonus die on some situations or automatic success on others.
Sometimes you might need to have secret to even try.
It is not necessary to have several status secrets if all apply, just choose most describing then (Wealthy king would might be Nobility where rich baron might be Wealthy).
However, if these are unrelated to each other, for example character is vampiric nobility and french political leader, it might make sense to have both.

\cols{
\StatusSecret{}
}


\textbf{Culture secrets:}

\textbf{Culture secrets} relate to specific culture.
To learn these you typically need teacher.
These are needed to not seem foreing in those cultures.
In general sense, languages are acquired through this but some basic knowledge on these can be acquired without need to buy secrets.
For example, it might be typical that people in some culture know several languages.
Then person with that background would also know several languages.
In almost every situation, two person who meet are able to communicate with each other.
In addition, two people from foreign culture are able to communicate with each other without outsider understanding (though there might be exeptions).
With these secret, you do not need to use \AbilityNameDeceit{} on people so that they would believe you to be one of them.

\cols{
\CultureSecret{}
}

\textbf{Ability related secrets:}

\textbf{Ability related secrets} augment uses of some abilities. 
These do not perform miracles instead allow user to utize their abilities.
Many of these secrets will take time and perhaps money to use.

\cols{
\AbilitySecret{}
}


\pagebreak

\textbf{Vampiric secrets:}

\textbf{Vampiric secrets} represent powers vampires posses.

\textbf{Vampiric strengths } is vigor based ability. 
Typically secrets are used in combat but might have uses also elsewhere.
If used in combat, vampire is then considered as a Monster.


\cols{
\VampiricStrengthSecret{}
}

\textbf{Vampiric mysticism } is instinct based ability.
These secret provide access to some reality shaping abilities vampires can posses.
These are strong and difficult to use.


\cols{
\VampiricMysticismSecret{}
}

\textbf{Vampiric mind control }

\cols{
\VampiricMindControlSecret{}
}

\pagebreak

%% KEYS %%

\textbf{Key}

Most important and difficult key is key of humanity. It consist of several steps where you can lose humanity slowly. 
Each vampire has set of principles.
More you have, more easily you can function in society.
Maximum number is 7 and minimum is 1.
If all principles are lost, vampire loses its mind.
As more and more principles are lost vampire becomes more obsessive on the remaining ones.
It is possible to develop other principles but whenever principle is lost it cannot be replaced.


% OTA uusi "vampiric curse"
\begin{itemize}
\setlength\itemsep{-2pt}
\item[7] normal vampiric curses still apply
\item[6-] it costs single point from any pool to wake up. If you do can't, you slumber
\item[5-] it costs 1 Vigor to appear as a living, to have bodily functions as a living
\item[4-] you acquire some mental illness
\item[3-] you are unable to eat regular food or drink
\item[2-] you cannot consume animal blood
\item[1] you cannot spend Vigor anymore to appear as a human.
\end{itemize}

% kerrallaan saa olla enintään kolme, toimivat kuin avaimet, mutta ilmaisia
Example principles are:
\begin{itemize}
\setlength\itemsep{-2pt}
\item Kindness towards others
\item Do onto others as you would have them to do to you
\item Justice and fairness
\item Right of everyone to decide their own fate
\item Right for everyone to their own bodies
\item Respect for human life
\item Integrity of your own body
\end{itemize}

You can write your own. These can also be changed between sessions. However, if one is lost it cannot be replaced. It will be marked as broken principle and it is permanent loss.

Key of Humanity is linked with these. As these describe principles, Key of Humanity is how these can be lost.

\KeyHumanity{}

\KeyNameHumanity{} speaks of remorse. If you break a principle you can:
\begin{itemize}
\item[a.] Buyoff principle. Cross over one principle that you broke with an action, take 10 experience. Vampire will always have \KeyNameHumanity{}
\item[b.] Feel remorse. Take Reason damage for each principle you broke. Each broken principle can contribute between 1 - 3 points of damage. If damage would be larger than largest damage track remorse is impossible.
\end{itemize}

If principles are changed it means that the new thing is actually the real principle and has always been, other has been self deception.
If vampire has broken new principle he has to feel remorse from each of those missteps.
Braking principle should be informed decision from players.
It should not be clear accident.
However, it might be something that player did not want to happen.

\textit{For example, new vampire is unable to control his bloodlust. He therfore tries to drink his beloved forcefully. This is does not respect right that everyone has to their own bodies or respect for life. However as this would be unintentional, it is deemed that first contributes 2 point and second only 1. This means that if this happens, it either cause buyoff or remorse worth of 3 points.
Player has no ability to fight against this kind of bloodlust knowing what the stake (in addition of the beloved).}

\pagebreak
\textbf{Vampiric keys}
These are optional keys but usually required. Without any following penalties apply:
\begin{itemize}
\setlength\itemsep{-2pt}
\item You have natural instinct for blood. Whenever you see blood and do not feed some, take one instinct damage.
\item Existance feels to be without meaning. It costs one from Instinct pool and Reason pool to rise on morning (in addition to any other costs)
\end{itemize}
These are both discourage and to relate what happens if these urges are not directed elsewhere.
These keys shown are just some possible keys.
Important part is that they relate to vampiric existance.

\cols{
\VampiricKey{}
}

%% POOLS %%
\pagebreak
Pools are normal, expect for refershing condition:

\begin{itemize}
\item[\textbf{Vigor}] Drink human blood directly from vein or drink animal blood and do some physical enjoyment. This can be hard partying, doing drugs, sparring or anything that involves someone else.
\item[\textbf{Instinct}] Hunt for blood in social setting or interact with your kind in social setting.
\item[\textbf{Reason}] Try to understand your existence. Alternatively, drink blood in intellectually stimulating interaction such as poetry recital, a late-night philosophical debate or a chess match.
\end{itemize}
}

\tableofcontents
\pagebreak

\section{Terms}\label{sec:terms}
\section{Introduction}\label{sec:introduction}
	\subsection{About this book}\label{ssec:about}
		\blindtext
\pagebreak

	\subsection{Roleplaying}\label{ssec:roleplaying}
		\blindtext
\pagebreak

	\subsection{Vampires}\label{ssec:vampires}
		\blindtext
\pagebreak

\section{System}\label{sec:System}
	\subsection{Overview}\label{ssec:overview}
		\blindtext
\pagebreak

	\subsection{Keys}\label{ssec:keys}
		\blindtext
\pagebreak

	\subsection{Abilities}\label{ssec:abilities}
		\blindtext
\pagebreak

	\subsection{Pools}\label{ssec:pools}
		\blindtext
\pagebreak

	\subsection{Secrets}\label{ssec:secrets}
		\blindtext
\pagebreak

	\subsection{Challenges}\label{ssec:challenges} % Tämä on perus tilanteisiin joissa heitetään noppaa
		\blindtext
\pagebreak

	\subsection{Conflict}\label{ssec:conflict} % myös extended
		\blindtext
\pagebreak

\section{Character}
	\subsection{Character creation}\label{ssec:creation}
		\blindtext
\pagebreak

	\subsection{Mortal}\label{ssec:mortal}
		Mortal characters are good beginning point for any character. 
Mortals in world filled with vampires can be really interesting characters.
These characters can be later updated to vampires, hunters, or thralls as story progresses.

Mortals do not possess unique abilities but can posses unique perspective.
Playing mortal can be great horror experience.
Large amount of vampire fiction is about facing superhuman threat that are out for your blood.

Mortal player characters now typically very little about vampires but more than their peers.
This knowledge puts them them into odds with both vampire and mortal worlds.
Other mortals consider those that know about vampires to be crazy and thus dangerous.
Vampires consider those characters to be nuicanse and possible threat.
However, even though character might know this they cannot return to their previous life, where they were safe as truth is large burden.

Mortal is somehow trying to interact with vampires.
They might have caused death of loved one, or might be mysterious love interest.
Whatever they think about vampires, they are problem.
As they do not know rules that vampires place, they might accidentally release vampire secrets to public.
Worst case might be, that existance of vampires is revealed.
Even asking questions, might be enought to incite violence.

To police, this seems to be stalking and perhaps even attempt on murder.
Even those that mortal convinces, are likely to become ostracised.
What makes things worse, are those vampiric abilities that alter memory, which might start person to question their own sanity.
Also some people these player characters talk to, might be thralls under vampire influence.
Thralls would only try futher isolation of the unwanted individual.


\pagebreak

	\subsection{Vampire}\label{ssec:vampire} % Vampiric keys(mandatory), Humanity and curses(nämä linkataa valittaessa yhteen)
		Vampire player characters are discussed in this chapter.

\subsubsection{Vampiric curses}

As vampire legends go, there are several.
It is up to each group to choose what kind creatures vampires are.
It is commonly known, that vampires have told several legends about themselves.
Some have done this to frighten, hurt other vampires, out of self loathing, or to misdirect.
These are represented with vampiric curses.

There are two ways of acquiring vampiric curses.
All vampires gain some set of curses when they shed mortality and become vampires.
This is not players choice, as it is more of what world does to them.
Other way is to lose \humanity{}, as each \humanity{} is paired with some curse.
That curse becomes part of vampires existance, if they fail to follow that ideal.
This is how vampires become more mosterous as time progresses, and struggle for them to keep their humanity.
List of vampiric curses can be found from \ref{ssec:curses}.

When creating vampire character, in addition to normal process, choose at least one key to represent your vampiric nature.
In addition, you need to choose at least one \humanity{} / vampiric curse pair.
Vampiric curse is not active as long you keept your \humanity{} intact.
If you brake against your \humanity{}, you take Instinct damage depending on bad your violation is.
This represent regret vampire feels for their action.
If vampire does not feel regret (players choice), then that \humanity{} is lost forever.
This can be treated as normal key buyoff.
Vampire should have 1-3 \humanities{}, getting new ones without spending advantages on them.
There is limit to this, you can have total of 7 different \humanities{} during \campaign{}.
Losing last one of these, means that there is nothing human left of the vampire and it has become pure monster.
This can be considered death of the character but player might be allowed to play to end of the current \session{}, showing what kind of monster vampire has become.

\cols{
\CurseSecret{}
}

\subsubsection{Vampiric secrets}
Vampires posses many mystical abilities.
These are represented with special secrets.
Each vampire secret has tiers.
You need to have lower tier power before you can gain higher tier power.
If you use secret in conflict, you can gain bonus dies.
You can gain benefit of maximum of three die per secret used.
It is also possible that using these abilities evade conflict or give vampire enough leverage challenge otherwise insurmountable threat.
It is very common that mortal might not be able to stand against vampire if they do not posses some special skills to face the vampire.
Activating powers give bonus die equal hightest tier of the power.
You can activate power several time for same conflict, but still the maximum amount can be worth of three dies.

\textit{Example: 
Vampire with Virtus wants to drink from mortal. 
Vampire uses his Virtus and overpowers human who tries to resist.
There is no conflict as mortal has no leverge against Vampire.}

\textit{
Example: 
Vampire with Virtus fights against hunter.
Vampire uses his Virtus and human hunter has prepared for this fight.
There is conflict and both roll. Vampire chooses to use his Virtus(II) and uses two from their Vigor pool in order to get 3 bonus die.}

Some secrets might be inherited through the blood and come easily.
Others should not come easily to vampire.
There should always be reason why vampire is able to learn the secret.
Each secret has some requirement for learning but these should be considered as examples.
This is learning that happens by experiencing something that allows vampire to understand something deep hidden inside them.
In addition to learning by experiencing some drastic moment, vampire might try to find a teacher.
Teach ability might be used to allow vampire to learn secret that teacher possess.
If teacher is not found, vampire might start quest for the secret.
Notice, that this means that even though some secret (e.g. Haruspicy(III)) requirement of some other secret (e.g. Haruspicy(II)) also other methods might be used to acquire it.

\pagebreak
\cols{
\SecretVampiricBase{}
\VampiricAbility{}
\VampiricSecret{}
\VampiricKey{}
}


	\subsection{Elder}\label{ssec:elder} % Vampyyri, joka on niin muinainen, että ei voi esittää samanlaisena kuin tavallisen, esim. 
		\blindtext
\pagebreak

	\subsection{Thrall}\label{ssec:thrall}
		\blindtext
\pagebreak

	\subsection{Hunter}\label{ssec:hunter}
		\blindtext
\pagebreak

\section{World}
	\subsection{Vampiric curses} \label{ssec:curses} % choose mandatory curses
		\blindtext
\pagebreak

	\subsection{Vampiric world} \label{ssec:vampire_world}
		\blindtext
\pagebreak

	\subsection{Time passes} \label{ssec:time_passes}
		Common theme that people want to explore with immortality, is passing of time.
This chapter is dedicated to help to explore, when months, years, or centuries pass, while vampire characters remain mostly unchanged. % or stagnant?
This is called time skip, meaning that we move our focus to more interesting time.
Not every game needs to have the passing of time.
During time skip, characters might, for example, learn new skills, consolidate power, heal wounds, build friendships, or make enemies.

First step is to decide that spot is good for doing the time skip.
It might come naturally but normally it requires that everyone on the table knows that there is going to be a time skip.
Good spot is after something major or intense has been just resolved.
The time skip should happen when there is still things ongoing but not immediately at least for every character.
It is also totally acceptable that someone asks that should we start a time skip, and someone respons not yet. % TODO: example?

After deciding that it is good time for a time skip, next it should be decided where and when time skip will take a place.
You can be open discussion but it can also be more interesting to let players make this decision. % erottelu player storyteller?
One inspirational way to do it, is to have one player to make decision on time, 
	and other player makes decision on where characters meet.
When deciding time, player can give more details on time, general stuff that everyone in the world knows.
When deciding location, description of scene and reason why characters meet there should be given.

\textit{First player want that where time skip ends is in 1806 AD. They want to specify that this is date is important as inquisition is abolished.} % TODO: better example?
\textit{Second player want that play will continue in Rome. They tell, that vampires are gathering there, as vampire ruler of Rome is going to make important declaration.}

Now players know how much time is covered with the time skip.
It might be months, or it might be years.
Vampires do not need to just be passive during this time.
Players describe what their characters were doing during this time.
This does not need to be extremely detailed, details can be figured out later.
Only important things should be outlined during this time.
I would recommend just specifying one thing but this is not strict sometimes there are several important things.
These typically cause skill check for each player character. % TODO: skill check?
% TODO: add example

As years pass, characters can also progress.
This is represented with gaining new secrets (Section~\ref{ssec:secrets}).
Normal rules of acquiring secrets still apply.
Secret gaining:
\begin{itemize}
\item[Month] 0 secret
\item[Year] 2 secret
\item[Decade] 4 secret
\item[Century] 6 secret
\item[Millenia] 8 secret
\end{itemize}
If people are not certain what secrets to buy, these can also be left undecided, and chosen when they would come up in story.
When these are chosen after time skip, it would be good to hear story, where character learned this secret
   \textit{(e.g. When Isabella was in France hiding from wrath of the bishop, she learned to synergize her Vagrancy with other skills) }.
Main purpose of this rule is to not rush people to make bad decisions.
Some secrets might difficult to acquire, and should be done immediatetly, and might require explanation how those are gained.
   \textit{(e.g. Isabella wanted to learn how to read people minds. That is reason why she went to the bishop, only person she knew that would accept money as payment ) }.
Details should are decided on table.



\comment{
%\AllSecrets{}

\begin{adjustwidth}{-1.25cm}{-1.25cm}
\begin{multicols} {3}
%\AllSecret{}
%\AllKey{}
\end{multicols}
\end{adjustwidth}
\section{Termist\"o}
\section{Esittely}
	\subsection{Kirjasta}
		\blindtext
	\subsection{Vampyyreista} % Vampiric curses, secrets
		\blindtext
	\subsection{Roolipelaamisesta}
		\blindtext
	\subsection{Terminologia}
		\blindtext
\section{Maailma}
	\subsection{Vampyyrien merkitys}
	\subsection{Vampyyrien ominaisuudet}
	\subsection{Vampyyri mytologian luonti}
\section{Pelaaminen}
	\subsection{Yleiskuva}
	\subsection{Pelin kulku}
	\subsection{Kyvyt}
	\subsection{Avaimet}
	\subsection{Salaisuudet}
	\subsection{Kykytesti}
	\subsection{Konflikti}
\section{Hahmo}
	\subsubsection{Perus Kuolevainen}
	\subsubsection{Vampyyri}
	\subsubsection{Vanha vampyyri}
	\subsubsection{Muinainen vampyyri}
	\subsubsection{Orja(Thrall) ja Metsästäjä}
	\subsection{Hahmolomake}
	\subsection{Hahmonluonti}
\section{Pelisessio}
	\subsection{Kykytesti}
	\subsection{Konflikti}
	\subsection{Valineet}
	\subsection{Ylosnousemus}
\section{Mekaniikka}
	\subsection{Kyvyt}
		\subsubsection{Innate Abilities}
			\paragraph{Endure}
			\paragraph{React}
			\paragraph{Resist}
		\subsubsection{Vampric Abilities}
			\paragraph{Quell the Beast}
			\paragraph{Strength}
			\paragraph{Mysticism}
			\paragraph{Domination}
		\subsubsection{Combat Abilities (Against)}
			\paragraph{Vampires}
			\paragraph{Hunters}
			\paragraph{War}
			\paragraph{Ranged}
			\paragraph{Brawl}
	\subsection{Avaimet}
		\subsubsection{Ihmisyys}
		\subsubsection{Vampyyrinen}
		\subsubsection{Yleiset}
	\subsection{Derangements}
	\subsection{Aikakausi}
	\subsection{Bloodline}
}
\end{document}

\newcommand{\topic}[1]{}
